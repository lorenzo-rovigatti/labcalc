%! TEX program = luatex
\documentclass[11pt]{article}
\usepackage{textcomp}
\usepackage{graphicx,wasysym, mdframed,xcolor,gensymb,verbatim}
\usepackage{color}
\usepackage{floatflt}
\usepackage[italian]{babel}
%T1 garantisce la visualizzazione corretta del font  FiraMono usato per i codici
%con T1 i caratteri sono rappresentati a 8 bit e quindi si hanno a disposizione 256 glyphs
%la scelta migliore sare TU (unicode) ma questa � supportata solo da XeTeX e LuaTeX
\input{listings_styles.tex}
\newcommand{\voto}[1]{[\textbf{#1} punti]}
\def\cmu{\mbox{cm$^{-1}$}}
\def\half{\frac{1}{2}}

\voffset -2cm
\hoffset -2.5cm
%\marginparwidth 0cm
\textheight 22cm
\textwidth 17cm
%\oddsidemargin  0.2cm                                                                                         
%\evensidemargin 0.4cm                                                                                         
\parindent=0pt

\begin{document}
\pagestyle{empty}
\begin{center}
{\Large \bf  Laboratorio di Calcolo per Fisici, Prima esercitazione\\[2mm]}
{\large Canale A-C 2019-2020}
\end{center}
\vspace{4mm}
\begin{mdframed}[backgroundcolor=panna]
  Lo scopo della prima esercitazione di laboratorio \`e di introdurre gli strumenti di base che verranno usati nel corso delle successive esercitazioni:
  la {\em shell}, l'editor di testo, il compilatore {\em gcc}, e {\em python\/} (tramite le librerie
  {\em matplotlib\/} e {\em numpy\/}) per la grafica.
\\
% Sulla pagina web del corso
% ({\em https://lboeri.wordpress.com/teaching/labcalc/ex/})
% sono disponibili dei tutorial pi\`u avanzati su ciascuno di questi argomenti.
\end{mdframed}
%\vspace{1mm}
%
%
\hrule
\vspace{2mm}
\textbf{$\RHD$ Prima parte (obbligatoria)} 
\begin{enumerate}
%\item  Se stai facendo l'esercitazione in presenza, fai  il {\em login\/} sulla propria macchina {\em Unix\/} utilizzando lo {\em userid\/} \texttt{lcgnxx}, 
%dove $xx$ \`e il tuo numero utente. 
\item Aprire una finestra di {\em terminale}.
\item Creare una cartella  \texttt{EX1}.
\item Entrare nella cartella \texttt{EX1} aprire con l'editor di testo il file \texttt{temp.c}, e digitare il listato sottostante. Salvare il contenuto del file.
\item Compilare il programma in c digitando sul terminale:
\texttt{gcc temp.c -o temp.o}
\item Eseguire il file {\em temp.o\/} digitando \texttt{./temp.o}
\item Inserire i dati richiesti dal programma; il programma \`e un semplice convertitore di temperature da gradi Celsius a gradi Fahrenheit.
\end{enumerate}
\begin{lstlisting}[caption={Programma \texttt{temp.c}},language=c]
#include <stdio.h>
int main()
{
  double tc, tf, conv, offset;
  conv = 5./9.;
  offset = 32.;
  printf("Inserisci la temperatura in gradi Celsius \n");
  scanf("%lf", &tc);
  tf = tc/conv + offset;
  printf("La temperatura in gradi Fahrenheit vale %5.2f gradi\n",tf);
}
\end{lstlisting}
\newpage
\textbf{$\RHD$ Seconda parte (obbligatoria)}\\
\vspace{2mm}
\textbf{Esercizio 1: Lanciare l'interprete python}\\
digitare sul vostro terminale il comando
\vspace{2mm}
\begin{lstlisting}[language=Python]
python3
\end{lstlisting}
\vspace{2mm}
e premere invio. Noterete che sul terminale compariranno tre freccette $>>>$ . Digitare ora il comando python
\vspace{2mm}
\begin{lstlisting}[language=Python]
print("Ciao Mondo")
\end{lstlisting}
\vspace{2mm}
e premere invio. Osservate il risultato.
Per chiudere l'interprete python digitate il comando
\vspace{2mm}
\begin{lstlisting}[language=Python]
exit()
\end{lstlisting}
\vspace{2mm}

Un altro modo \`{e} quello di creare uno script in  python da far interpretare. Aprire un editor di testo e digitare 
\vspace{2mm}
\begin{lstlisting}[language=Python]
#lo script scrive su terminale la frase "Ciao Mondo"
print("Ciao Mondo")
\end{lstlisting}
\vspace{2mm}
salvate il contenuto del file e nominate il file \texttt{ex1.py}. Chiudete l'editor di testo.
Nel terminale scrivete il comando
\vspace{2mm}
\begin{lstlisting}[language=Python]
python3 ex1.py
\end{lstlisting}
\vspace{2mm}
e premete invio. Osservate il risultato.
\vspace{2mm}
\hrule
\vspace{4mm}
\textbf{Esercizio 1.2: L'indentazione in python}\\
Questo esercizio mostra come l'indentazione sia obbligatoria in python quando un insieme di istruzioni devono essere raggruppate. Essa \`{e} l'equivalente dell'uso delle parentesi graffe in C.
Aprire un editor di testo e digitare 
\vspace{2mm}
\begin{lstlisting}[language=Python]
for i in "ciao":
    print(i)
\end{lstlisting}
\vspace{2mm}
dove nella seconda linea avete lasciato $4$ spazi vuoti dall'inizio della riga (indentazione).Salvate il contenuto del file e nominate il file \texttt{ex1-2.py}. Chiudete l'editor di testo.
Nel terminale scrivere il comando
\vspace{2mm}
\begin{lstlisting}[language=Python]
python3 ex1-2.py
\end{lstlisting}
\vspace{2mm}
e premete invio. Osservare il risultato.
Adesso aprire \texttt{ex1-2.py} e togliere i primi $4$ spazi nella seconda linea del codice. Salvare e lanciare lo script come fatto nel primo caso. Osservare il risultato.
\vspace{2mm}
\hrule
\vspace{4mm}
\textbf{Esercizio 2: I moduli matplotlib e pyplot}\\
Questo esercizio mostra come importare il modulo pyplot in matplotlib e come graficare dei dati.
Aprire un editor di testo e digitare
\vspace{2mm}
\begin{lstlisting}[caption= ex2.py, language=Python]
#un primo grafico in python
import matplotlib.pyplot as plt
plt.plot([-4, -3, -2, -1, 0, 1, 2, 3, 4])
plt.show()
\end{lstlisting}
\vspace{2mm}
Salvare lo script con il nome ex2.py e lanciare lo script come visto in precendenza. Esplorare le opzioni della toolbar del grafico.
Modificare ora lo script \texttt{ex2.py} nel seguente modo
\vspace{2mm}
\begin{lstlisting}[caption= ex2.py modificato, language=Python]
#un primo grafico in python
import matplotlib.pyplot as plt
plt.plot([-4, -3, -2, -1, 0, 1, 2, 3, 4], [16, 9, 4, 1, 0, 1, 4, 9, 16])
plt.show()
\end{lstlisting}
\vspace{2mm}
salvare il file e lanciare lo script. Osservare il risultato.
\vspace{2mm}
\hrule
\vspace{4mm}
\textbf{Esercizio 3: Il modulo numpy}\\
Questo esercizio mostra come  il modulo numpy permetta di graficare funzioni matematiche in maniera agile rispetto ai casi visti prima.
Aprire un editor di testo e digitare le seguenti istruzioni:
\vspace{2mm}
\begin{lstlisting}[caption= ex3.py, language=Python]
#lo script grafica una parabola
import matplotlib.pyplot as plt
import numpy as np
x=np.linspace(-4, 4, 50, endpoint=True)
y=x*x 
plt.plot(x,y,'o')
plt.show()
\end{lstlisting}
\vspace{2mm}
salvare il file  con il nome ex3.py e lanciare lo script. Osservare il risultato e successivamente fare delle prove modificando il numero dei punti e gli estremi della curva.
\vspace{2mm}
\hrule
\vspace{4mm}
\textbf{Esercizio 3-1: lettura dei dati da file}\\
Questo esercizio mostra come importare dei dati contenuti in un file di testo per graficarlo. Aprire un editor di testo e copiare le seguenti colonne all'interno del nuovo file:
\vspace{2mm}
\begin{lstlisting}[caption= myfile.dat, language=Python]
#x	#y
-4	16
-3	 9
-2	 4
-1 	 1
 0	 0
 1	 1
 2	 4
 3	 9	
 4	16
\end{lstlisting}
\vspace{2mm}
nominare il file myfile.dat .
Aprire un secondo file con un editor di testo e copiare al suo interno il seguente script:
\vspace{2mm}
\begin{lstlisting}[caption= ex3-1.py, language=Python]
#un primo grafico in python
import matplotlib.pyplot as plt
import numpy as np
x,y=np.loadtxt('myfile.dat',unpack=True, usecols=(0,1))
plt.plot(x,y,'-o')
plt.savefig('grafico.png')   #comando per salvare il grafico con il nome grafico.png
plt.show()
\end{lstlisting}
\vspace{2mm}
salvare il file con il nome \texttt{ex3-1.py} e lanciare lo script. Osservate il risultato.
Aggiungete ora una terza colonna nel file myfile.dat.
\vspace{2mm}
\begin{lstlisting}[caption= myfile.dat modificato, language=Python]
#x	#y	#z
-4	16	-64
-3	 9	-27	
-2	 4	-8
-1 	 1	-1
 0	 0	0
 1	 1	1
 2	 4	8	
 3	 9	27
 4	16	64
\end{lstlisting}
\vspace{2mm}
Modificare lo script \texttt{ex3-1.py} in modo da leggere anche la terza colonna come mostrato di seguito
\vspace{2mm}
\begin{lstlisting}[caption= ex3-1.py modificato, language=Python]
#un primo grafico in python
import matplotlib.pyplot as plt
import numpy as np
x,y,z=np.loadtxt('myfile.dat',unpack=True, usecols=(0,1,2))
plt.plot(x,z,'-o')
plt.savefig('grafico1.png') 
plt.show()
\end{lstlisting}
\vspace{2mm}
Apportare le seguenti modifiche: (1) Modificare il codice in modo da prendere in input solo la seconda e terza colonna di myfile.dat (2) Modificare il codice in modo da leggere tutte e tre le colonne da myfile.dat e graficare sia x vs y che y vs z.
Il secondo grafico si ottiene aggiungendo una seconda istruzione plt.plot(...) subito dopo la prima.
\vspace{2mm}
\hrule
\vspace{4mm}
\textbf{Esercizio 4: personalizzazione dei grafici}
Con questo esempio vedremo alcune opzioni per migliorare l'estestica di un grafico. Copiare in un file di testo il seguente script python (non c'e' bisogno di copiare i commenti):
\vspace{2mm}
\begin{lstlisting}[caption= ex4.py, language=Python]
#questo programma grafica il cos(x) tra [-2pi;2pi]
#sia pi (pigreco) che cos() (funzione coseno) fanno parte del modulo numpy
#Vanno quindi precedute da np.
import matplotlib.pyplot as plt
import numpy as np
x=np.linspace(-2*np.pi,2*np.pi,100,endpoint=True) 
y=np.cos(x) 
plt.plot(x,y,linewidth=4,color='b',linestyle='-',marker='o')
plt.savefig('cos.png')
plt.show()
\end{lstlisting}
\vspace{2mm}
salvare e nominare il file ex4.py. Lanciare lo script e osservare il risultato. Modificare i parametri della funzione plt.plot()  usando le opzioni in Fig.\ref{fig:fig1} (in fondo al testo).
\vspace{2mm}
\hrule
\vspace{4mm}
\textbf{Esercizio 4-1: modificare l'intervallo del grafico}
Modificate lo script precedente (ex4.py) aggiungendo la seguente riga di codice prima dell'istruzione per fare il grafico
(non c'e' bisogno di copiare i commenti)
\vspace{2mm}
\begin{lstlisting}[caption= ex4.py modificato, language=Python]
plt.xlim(-np.pi,np.pi) #restringe l'intervallo su x tra -pigreco e pigreco
plt.ylim(-2.,2.)       #restringe l'intervallo su y tra  -2  e 2
\end{lstlisting}
\vspace{2mm}
modificare i campi delle due istruzioni per variare l'intervallo dell'asse x e y.
\vspace{2mm}
\hrule
\vspace{4mm}
\newpage
\textbf{Esercizio 5: Etichette degli assi, titolo  e legenda}\\
In questo esempio vedremo come aggiungere delle etichette sull'asse x e y, dare un titolo al grafico e graficare la legenda.
Copiare lo script \texttt{ex4.py} in un nuovo file di testo  e modificarlo come mostrato di seguito (non c'e' bisogno di copiare i commenti):
\vspace{2mm}
\begin{lstlisting}[caption= ex4.py modificato, language=Python]
#questo programma grafica il cos(x) tra [-2pi;2pi]
import matplotlib.pyplot as plt
import numpy as np
x=np.linspace(-2*np.pi,2*np.pi,100,endpoint=True)
y=np.cos(x)
plt.title('funzione coseno')     #titolo del grafico               
plt.xlabel('asse x')             #etichetta asse x
plt.ylabel('asse y')             #etichetta assey
plt.plot(x,y,'b-',label='cos(x)',linewidth=3) #il campo label rappresenta la legenda
plt.legend()                     #visualizza la legenda nel grafico (da mettere sempre dopo plt.plot())
plt.savefig('cos.png')
plt.show()
\end{lstlisting}
\vspace{2mm}
chiamare il file creato \texttt{ex5.py}. Lanciare lo script e osservare il risultato. Provare a modificare il titolo, la legenda e i nomi degli assi. (Opzionale:) Graficare assieme al coseno anche sin(x). Aggiungere la legenda anche per questo seconda curva.
\vspace{2mm}
\hrule
\vspace{4mm}
\textbf{$\RHD$ Esercizio (facoltativo)} 
Si grafichi funzione $y(x)=x3$ con x [$-5.25$, $5.25$] (con l'estremo destro incluso). Il grafico deve contenere una legenda, un titolo  e delle etichette sull'asse $x$ e $y$. Il numero di punti con cui la curva deve essere rappresentata \`{e} $80$. Il range lungo l'asse x del grafico deve essere [$-5.5$, $5.5$] mentre sull'asse y [$-200.0$, $200.0$]. Giocare con gli stili, colori, simboli ecc. e salvare il grafico come un file .png dal nome cubic.png
\vspace{2mm}
\hrule
\vspace{4mm}
\textbf{$\RHD$ Esercizio (facoltativo)} 
%
\begin{enumerate}
\item Eseguire il programma \texttt{temp.x} quattro o pi\`u volte, con valori di input diversi, e creare un file di testo chiamato \texttt{temp.dat} con due colonne,
che contenga i valori di input e di output --- temperatura in Celsius (Tc) e temperatura in Fahrenheit (Tf) (dal valore piu' piccolo al piu' grande).
%\item 
%Aprire il programma di grafica \texttt{gnuplot} digitando nella shell il
%comando \texttt{gnuplot}.
\item Graficare i dati contenuti nel file \texttt{temp.dat} creando uno script python che crei un grafico con un titolo e con nomi sull'asse $x$ e $y$:\\
\item graficare assieme ai dati del file  \texttt{temp.dat} la retta li interpola, conoscendo la relazione che lega la temperatura in Celsius e Fahrenheit
\end{enumerate}
%Utilizzando l'{\em help} di {\em gnuplot}, che si invoca con il comando \texttt{help},


\begin{figure}[!h]
\includegraphics[width=1.0\columnwidth]{./options.png}
\caption{opzioni per personalizzare lo stile di un grafico}
\label{fig:fig1}
\end{figure}

\end{document}
