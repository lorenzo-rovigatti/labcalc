%! TEX program = luatex
\documentclass[11pt]{article}
\usepackage{graphicx,wasysym, mdframed,xcolor,gensymb,verbatim}
\usepackage{color}
\usepackage{floatflt}
\usepackage[italian]{babel}
\input{custom.tex}
\def\cmu{\mbox{cm$^{-1}$}}
\def\half{\frac{1}{2}}

\voffset -2cm
\hoffset -2.5cm
%\marginparwidth 0cm
\textheight 22cm
\textwidth 17cm
%\oddsidemargin  0.2cm                                                                                         
%\evensidemargin 0.4cm                                                                                         
\parindent 0pt

\begin{document}
\pagestyle{empty}

\begin{center}
{\Large \bf  Laboratorio di Calcolo per Fisici,\\ Seconda Esercitazione Valutata, 11/12/2017 \\[2mm]}
{\large Canale \canale, Docente: \docente}
\end{center}
\vspace{4mm}

\begin{mdframed}[backgroundcolor=gray!10]
Lo scopo della seconda esercitazione valutata \`e scrivere un programma che simuli il gioco della {\em morra cinese}. \\
Per svolgere l'esercitazione avrete 3 ore; sono concessi libri di testo e appunti
ed \`e ammesso discutere la soluzione con il proprio compagno di gruppo
(a bassa voce), ma non con gli altri gruppi.
\\
{\bf L'uso di cellulari e tablet non \`e ammesso, pena l'annullamento del compito.}
\\
Il programma va scritto e salvato esclusivamente sul server del laboratorio,
utilizzando lo user-id corrispondente al vostro gruppo, in una cartella
di nome \texttt{EX9}, su un file di nome \texttt{morra.c}. Per sicurezza inserite 
nelle prime righe del file due righe di commento contenenti il nome, cognome
e numero di matricola dei componenti del gruppo.
  \end{mdframed}
%\vspace{1mm}
%
%



\hrule
\vspace{2mm}
\textbf{$\RHD$ Esercizio:}
La morra cinese \`e un antico gioco che si svolge tra due persone come segue.
Ciascuno dei giocatori porta la mano dietro la testa e, ritmicamente, la riporta
rapidamente davanti a s\'e facendo un gesto convenzionale ({\em getto}). 
I gesti possono essere di tre tipi: il pungo chiuso rappresenta un sasso, la mano aperta un foglio di carta e la mano chiusa con l'indice e il medio stesi un paio di forbici. Si confrontano i segni di ciascuno dei due giocatori e si stabilisce chi vince secondo le regole che seguono:
\begin{itemize}
\item Il sasso vince sulle forbici. 
\item Le forbici vincono sulla carta.
\item La carta vince sul sasso.
\end{itemize}
Due segni uguali conducono a un pareggio e s'ignorano. Vince chi, dopo un numero prestabilito di getti ottiene il maggior numero di vittorie. Simulare una partita di morra cinese nel modo che segue.
\begin{enumerate}
\item Il programma chiede all'utente di quanti getti sar\`a composta la partita.
L'utente inserisce questo numero attraverso la tastiera.
\item Il programma controlla che il numero di getti introdotto sia positivo e minore o uguale a 20.
\item Si simula il comportamento di un giocatore attraverso una funzione che
restituisce con uguale probabilit\`a un numero intero che rappresenta il segno
fatto da ciascun giocatore.
\item  A una funzione si passano i segni del giocatore A e del giocatore B. La
funzione determina il vincitore in base alle regole sopra citate e restituisce
un valore che permette di discriminare chi ha vinto il getto o se il getto \`e finito in parit\`a. La scelta della maniera in cui questo si ottiene \`e lasciata
allo studente.
\item Per ciascun getto si determina il vincitore e si aggiunge un punto al vincitore, se esiste.
\item L'esito del singolo confronto dev'essere memorizzato, in qualche maniera,
in un opportuno array.
\item Al termine delle partita si decreta il vincitore e si stampa, sullo schermo,
la sequenza dei getti. Ci sar\`a una riga per ciascun getto,
in cui saranno riportate la scelta di ciascun giocatore ({\bf C}, {\bf F} o {\bf S}),
e l'esito di ciascun getto (vittoria di $A$ o $B$ o pareggio).

\end{enumerate}









\end{document}
