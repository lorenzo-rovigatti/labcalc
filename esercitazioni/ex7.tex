%! TEX program = luatex
\documentclass[11pt]{article}
\usepackage{textcomp}
\usepackage{graphicx,wasysym, mdframed,xcolor,gensymb,verbatim}
\usepackage{color}
\usepackage{floatflt}
%se si usa pdflatex:
%\usepackage[utf8]{inputenc}
%\usepackage[scaled=0.9]{FiraSans}
%
%i seguenti comandi funzionano con lualatex (si possono usare tutti i font di sistema come in vim o nel terminale!)
\usepackage{fontspec}
%\setmonofont{Inconsolatazi4}
%
% 0 OfficeCodePro 1=inconsolata 2=Hack
\ifcase 0 %font 0
\setmonofont[Scale=0.7,
  ItalicFont=OfficeCodePro-RegularItalic,
  BoldFont=OfficeCodePro-Bold,
  BoldItalicFont=OfficeCodePro-BoldItalic,
  UprightFont=OfficeCodePro-Regular]
  {OfficeCodePro}
\or% font 1
\setmonofont[Scale=0.7,
  ItalicFont=inconsolatalgcitalic,
  BoldFont=inconsolatalgcbold,
  BoldItalicFont=inconsolatalgcitalic,
  UprightFont=inconsolatalgc]
  {inconsolatagc}
\else% else
%
%imposto il font per listings (si possono anche indicare le varianti, i.e. italic, bold, bolditalic)
\setmonofont[Scale=0.7,
  ItalicFont=Hack-Italic,
  BoldFont=Hack-Bold,
  BoldItalicFont=Hack-BoldItalic,
  UprightFont=Hack-Regular]
  {Hack}
\fi
%
\usepackage[T1]{fontenc}
% un'alternativa a listingsutf8 è il pacchetto minted ma richiede una libreria python chiamata Pygments
\usepackage{listingsutf8}
\definecolor{verdeoliva}{rgb}{0.3,0.3,0}
\definecolor{grigio}{rgb}{0.5,0.5,0.5}
\definecolor{blumarino}{rgb}{0.0,0,0.5}
\definecolor{panna}{rgb}{0.98,0.98,0.94}
\def\lstlistingname{Listato}
\lstset{%
  %inputencoding=utf8,
  breaklines=true,
  %extendedchars=true,              % lets you use non-ASCII characters; for 8-bits encodings only, does not work with UTF-8
  %literate=%
  %       {á}{{\'a}}1
  %       {í}{{\'i}}1
  %       {é}{{\'e}}1
  %       {ý}{{\'y}}1
  %       {ú}{{\'u}}1
  %       {ó}{{\'o}}1,
  backgroundcolor=\color{panna},   % choose the background color; you must add \usepackage{color} or \usepackage{xcolor}; should come as last argument
% basicstyle=\footnotesize\ttfamily,
  basicstyle=\ttfamily,            % è selezionato all'inizio di ogni listing
  belowskip=-0.2\baselineskip,
% basicstyle=\footnotesize,        % the size of the fonts that are used for the code
  breakatwhitespace=false,         % sets if automatic breaks should only happen at whitespace
% breaklines=true,                 % sets automatic line breaking
  captionpos=b,                    % sets the caption-position to bottom
  commentstyle=\itshape\color{verdeoliva}, % comment style (nota: all'inizio del listing seleziona \ttfamily, perciò qui seleziona la variante italic)
% deletekeywords={},            % if you want to delete keywords from the given language
% escapeinside={\%*}{*)},          % if you want to add LaTeX within your code
% firstnumber=1000,                % start line enumeration with line 1000
  frame=single,	                   % adds a frame around the code
  keepspaces=true,                 % keeps spaces in text, useful for keeping indentation of code (possibly needs columns=flexible)
  keywordstyle=\bfseries\color{blue},       % keyword style
% language=Octave,                 % the language of the code
% morekeywords={*,},            % if you want to add more keywords to the set
  numbers=left,                    % where to put the line-numbers; possible values are (none, left, right)
  numbersep=5pt,                   % how far the line-numbers are from the code
  numberstyle=\tiny\color{grigio}, % the style that is used for the line-numbers
  rulecolor=\color{black},         % if not set, the frame-color may be changed on line-breaks within not-black text (e.g. comments (green here))
  showspaces=false,                % show spaces everywhere adding particular underscores; it overrides 'showstringspaces'
  showstringspaces=false,          % underline spaces within strings only
  showtabs=false,                  % show tabs within strings adding particular underscores
  stepnumber=1,                    % the step between two line-numbers. If it's 1, each line will be numbered
  stringstyle=\color{blumarino},   % string literal style
  tabsize=2,	                   % sets default tabsize to 2 spaces
  title=\lstname%                  % show the filename of files included with \lstinputlisting; also try caption instead of title
}

\input{custom.tex}
\usepackage[shortlabels]{enumitem}
\usepackage[italian]{babel}
\def\cmu{\mbox{cm$^{-1}$}}
\def\half{\frac{1}{2}}

\voffset -2cm
\hoffset -2.5cm
%\marginparwidth 0cm
\textheight 22cm
\textwidth 17cm
%\oddsidemargin  0.2cm                                                                                         
%\evensidemargin 0.4cm                                                                                         
\parindent 0pt

\begin{document}
\pagestyle{empty}

\begin{center}
{\Large \bf  Laboratorio di Calcolo per Fisici, Settima esercitazione\\[2mm]}
{\large Canale \canale, Docente: \docente}
\end{center}
\vspace{4mm}

\begin{mdframed}[backgroundcolor=gray!10]
  Lo scopo della settima esercitazione di laboratorio è di fare pratica con
gli argomenti appresi durante il corso in vista delle esercitazioni valutate;
l'argomento di questa esercitazione è la gestione di matrici $3 \times 3$.
  \end{mdframed}
%\vspace{1mm}
%
%



\hrule
\vspace{2mm}
Il determinante di una matrice quadrata $3 \times 3$ si pu\`{o} calcolare usando la {\em regola di Sarrus}, che consiste in quanto segue: dalla matrice A, con $N = 3$ righe 
e altrettante colonne, se ne ricava una $B$ con $N$ righe e $2N$ colonne in
cui la met\`{a} destra di $B$ \`{e} una copia esatta di $A$. Si calcolano quindi i prodotti $p$ degli elementi che si trovano
lungo tutte le $N$ diagonali che si possono costruire partendo dall' elemento in alto a sinistra e si sommano tra loro
algebricamente. Successivamente si calcolano gli $N$ prodotti degli elementi che si trovano lungo le $N$ diagonali che
si possono costruire in direzione opposta partendo dall' elemento in alto a destra di $B$. Questi prodotti si sommano
col segno cambiato a quanto ottenuto al passaggio precedente. Il risultato di questa somma \`{e} il determinante della
matrice.

Consideriamo, per esempio, la matrice seguente:
\[
A = \left(\begin{array}{ccc}
5 &8& 6\\
3 &5 &9\\
7 & 9 & 9
\end{array}
\right)
\]

Ricaviamo la matrice $B$ come
\[
B = \left(\begin{array}{cccccc}
5 & 8 & 6 &5 & 8 & 6\\
3 & 5 & 9 &3 &5  &9\\
7 & 9 & 9 &7 &9 & 9
\end{array}
\right)
\]

Le possibili diagonali dall'elemento in alto a sinistra formano i seguenti prodotti: $(5 \times 5 \times 9)= 225$;
$(8 \times 9 \times 7)= 504$; 
e $(6\times3\times9)=162$. Quelle in direzione opposta sono invece: 
$(6\times5\times7)=210$, $(8\times3\times9)=216$ e $(5\times9\times9)=405$.
\\
Dati questi valori si calcola
$\det A = 225 + 504 +162 - 210 - 216 -405= 60$ .

\vspace{2mm}
\hrule
\vspace{2mm}
\textbf{$\RHD$ Prima parte:}

\begin{enumerate}
\item Creare una cartella EX7 dove raccogliere il materiale dell'esercitazione
\item Creare un programma \texttt{sarrus.c} che calcoli il determinante di una matrice intera $3 \times 3$ 
utilizzando la formula di Sarrus. Il programma dovr\`{a}:
\begin{enumerate}
\item stampare un messaggio iniziale che spieghi all'utente che cosa fa il programma
\item offrire all'utente la possibilit\`{a} di inserire gli elementi della matrice da tastiera 
o di leggerli da file. Se si sceglie la seconda opzione viene richiesto il nome del file  da leggere 
\item scrivere una funzione chiamata \texttt{CalcSarrus} che 
\begin{itemize}
\item prenda in input la matrice
\item crei un array multidimensionale \texttt{B[3][6]} contente 
la matrice in input duplicata (come nell'esempio sopra)
\item calcoli i prodotti lungo le diagonali di B usando dei cicli
\item restituisca il valore del determinante
\end{itemize}
\item stampare la matrice in input sullo schermo in maniera leggibile
\item stampare su schermo il determinante della matrice in input calcolato attraverso la funzione \texttt{CalcSarrus}
\end{enumerate}
\end{enumerate}

\hrule
\vspace{2mm}
\textbf{$\RHD$ Seconda parte:}
\vspace{2mm}
\begin{enumerate}
\item
Aggiungere al programma una funzione \texttt{genvec} che generi un vettore di numeri casuali $\vec{x}=\left(x_1,x_2,x_3\right)$ 
compresi tra $[-1,1)$. Si dividano le tre componenti per il modulo del vettore e si controlli che il 
vettore risultante \`e un versore (ossia un vettore di modulo 1).
\item Aggiungere una funzione \texttt{product} che calcoli il risultato del prodotto: 
$A \vec{x}$ e lo salvi nel vettore $\vec{y}$.
\item Stampare su schermo le componenti del vettore $\vec{y}$
\item Controllare che, usando come input la matrice
\[
R = \left(\begin{array}{ccc}
 0 & 0 & 1\\
 0 & 1 & 0\\
-1 & 0 & 0
\end{array}
\right),
\]
Il prodotto $\vec{y}=A \vec{x}$ restituisca un vettore $\vec{y}$ di modulo 1.
\item Usando come matrice di input la matrice $R$ descritta sopra, generare tanti vettori $\vec{y}$ (almeno 100 usando un ciclo \texttt{for}) e scriverli su un file chiamato
\texttt{coordinate.dat} nel formato
y[0] y[1] y[2].\\
\item Usando python, graficare la prima e ultima colonna del file cos\`i generato e salvare il grafico nel file \texttt{coordinate.png}. Che cosa notate aumentando il numero di punti da graficare? Commentare il grafico in risposta.txt
\end{enumerate}

\begin{mdframed}[backgroundcolor=gray!10]
{\em Un piccolo aiuto:} Per passare un array multidimensionale a una funzione \`{e} sempre necessario specificare la seconda  dimensione dell'array in maniera esplicita sia in fase di dichiarazione che di definizione della funzione;
il {\em prototipo} della funzione \texttt{prodotto} \`{e} perci\`{o} qualcosa del tipo:
\\
\texttt{
void product(int matrice[][3], ecc... );}.

\end{mdframed}

\end{document}
