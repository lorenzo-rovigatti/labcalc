%! TEX program = luatex
\documentclass[11pt]{article}
\usepackage{textcomp}
\usepackage{graphicx,wasysym, mdframed,xcolor,gensymb,verbatim}
\usepackage{color}
\usepackage{floatflt}
\input{listings_styles.tex}
\newcommand{\persinfo}[3] {%
  \newcommand{\canale}{#1}
  \newcommand{\docente}{#2}
  \newcommand{\login}{#3}
}
\input{persinfo.tex}

\usepackage[shortlabels]{enumitem}
\usepackage[italian]{babel}
\def\cmu{\mbox{cm$^{-1}$}}
\def\half{\frac{1}{2}}

\voffset -2cm
\hoffset -2.5cm
%\marginparwidth 0cm
\textheight 22cm
\textwidth 17cm
%\oddsidemargin  0.2cm                                                                                         
%\evensidemargin 0.4cm                                                                                         
\parindent 0pt

\begin{document}
\pagestyle{empty}

\begin{center}
{\Large \bf  Laboratorio di Calcolo per Fisici, Settima esercitazione\\[2mm]}
{\large Canale \canale, Docente: \docente}
\end{center}
\vspace{4mm}

\begin{mdframed}[backgroundcolor=gray!10]
  Lo scopo della settima esercitazione di laboratorio è di fare pratica con
gli argomenti appresi durante il corso in vista delle esercitazioni valutate;
l'argomento di questa esercitazione è la gestione di matrici $3 \times 3$.
  \end{mdframed}
%\vspace{1mm}
%
%



\hrule
\vspace{2mm}
Il determinante di una matrice quadrata $3 \times 3$ si pu\`{o} calcolare usando la {\em regola di Sarrus}, che consiste in quanto segue: dalla matrice A, con $N = 3$ righe 
e altrettante colonne, se ne ricava una $B$ con $N$ righe e $2N$ colonne in
cui la met\`{a} destra di $B$ \`{e} una copia esatta di $A$. Si calcolano quindi i prodotti $p$ degli elementi che si trovano
lungo tutte le $N$ diagonali che si possono costruire partendo dall' elemento in alto a sinistra e si sommano tra loro
algebricamente. Successivamente si calcolano gli $N$ prodotti degli elementi che si trovano lungo le $N$ diagonali che
si possono costruire in direzione opposta partendo dall' elemento in alto a destra di $B$. Questi prodotti si sommano
col segno cambiato a quanto ottenuto al passaggio precedente. Il risultato di questa somma \`{e} il determinante della
matrice.

Consideriamo, per esempio, la matrice seguente:
\[
A = \left(\begin{array}{ccc}
5 &8& 6\\
3 &5 &9\\
7 & 9 & 9
\end{array}
\right)
\]

Ricaviamo la matrice $B$ come
\[
B = \left(\begin{array}{cccccc}
5 & 8 & 6 &5 & 8 & 6\\
3 & 5 & 9 &3 &5  &9\\
7 & 9 & 9 &7 &9 & 9
\end{array}
\right)
\]

Le possibili diagonali dall'elemento in alto a sinistra formano i seguenti prodotti: $(5 \times 5 \times 9)= 225$;
$(8 \times 9 \times 7)= 504$; 
e $(6\times3\times9)=162$. Quelle in direzione opposta sono invece: 
$(6\times5\times7)=210$, $(8\times3\times9)=216$ e $(5\times9\times9)=405$.
\\
Dati questi valori si calcola
$\det A = 225 + 504 +162 - 210 - 216 -405= 60$ .

\vspace{2mm}
\hrule
\vspace{2mm}
\textbf{$\RHD$ Prima parte:}

\begin{enumerate}
\item Creare una cartella EX7 dove raccogliere il materiale dell'esercitazione
\item Creare un programma \texttt{sarrus.c} che calcoli il determinante di una matrice intera $3 \times 3$ 
utilizzando la formula di Sarrus. Il programma dovr\`{a}:
\begin{enumerate}
\item stampare un messaggio iniziale che spieghi all'utente che cosa fa il programma
\item offrire all'utente la possibilit\`{a} di inserire gli elementi della matrice da tastiera 
o di leggerli da file. Se si sceglie la seconda opzione viene richiesto il nome del file  da leggere 
\item scrivere una funzione chiamata \texttt{CalcSarrus} che 
\begin{itemize}
\item prenda in input la matrice
\item crei un array multidimensionale \texttt{B[3][6]} contente 
la matrice in input duplicata (come nell'esempio sopra)
\item calcoli i prodotti lungo le diagonali di B usando dei cicli
\item restituisca il valore del determinante
\end{itemize}
\item stampare la matrice in input sullo schermo in maniera leggibile
\item stampare su schermo il determinante della matrice in input calcolato attraverso la funzione \texttt{CalcSarrus}
\end{enumerate}
\end{enumerate}

\hrule
\vspace{2mm}
\textbf{$\RHD$ Seconda parte:}
\vspace{2mm}
\begin{enumerate}
\item
Aggiungere al programma una funzione \texttt{genvec} che generi un vettore di numeri casuali $\vec{x}=\left(x_1,x_2,x_3\right)$ 
compresi tra $[-1,1)$. Si dividano le tre componenti per il modulo del vettore e si controlli che il 
vettore risultante \`e un versore (ossia un vettore di modulo 1).
\item Aggiungere una funzione \texttt{product} che calcoli il risultato del prodotto: 
$A \vec{x}$ e lo salvi nel vettore $\vec{y}$.
\item Stampare su schermo le componenti del vettore $\vec{y}$
\item Controllare che, usando come input la matrice
\[
R = \left(\begin{array}{ccc}
 0 & 0 & 1\\
 0 & 1 & 0\\
-1 & 0 & 0
\end{array}
\right),
\]
Il prodotto $\vec{y}=A \vec{x}$ restituisca un vettore $\vec{y}$ di modulo 1.
\item Usando come matrice di input la matrice $R$ descritta sopra, generare tanti vettori $\vec{y}$ (almeno 100 usando un ciclo \texttt{for}) e scriverli su un file chiamato
\texttt{coordinate.dat} nel formato
y[0] y[1] y[2].\\
\item Usando python, graficare la prima e ultima colonna del file cos\`i generato e salvare il grafico nel file \texttt{coordinate.png}. Che cosa notate aumentando il numero di punti da graficare? Commentare il grafico in risposta.txt
\end{enumerate}

\begin{mdframed}[backgroundcolor=gray!10]
{\em Un piccolo aiuto:} Per passare un array multidimensionale a una funzione \`{e} sempre necessario specificare la seconda  dimensione dell'array in maniera esplicita sia in fase di dichiarazione che di definizione della funzione;
il {\em prototipo} della funzione \texttt{prodotto} \`{e} perci\`{o} qualcosa del tipo:
\\
\texttt{
void product(int matrice[][3], ecc... );}.

\end{mdframed}

\end{document}
