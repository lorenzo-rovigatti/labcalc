%! TEX program = luatex
\documentclass[11pt]{article}
\usepackage{textcomp}
\usepackage{graphicx,wasysym, mdframed,xcolor,gensymb,verbatim}
\usepackage{color}
\usepackage{floatflt}
\usepackage[italian]{babel}
%se si usa pdflatex:
%\usepackage[utf8]{inputenc}
%\usepackage[scaled=0.9]{FiraSans}
%
%i seguenti comandi funzionano con lualatex (si possono usare tutti i font di sistema come in vim o nel terminale!)
\usepackage{fontspec}
%\setmonofont{Inconsolatazi4}
%
% 0 OfficeCodePro 1=inconsolata 2=Hack
\ifcase 0 %font 0
\setmonofont[Scale=0.7,
  ItalicFont=OfficeCodePro-RegularItalic,
  BoldFont=OfficeCodePro-Bold,
  BoldItalicFont=OfficeCodePro-BoldItalic,
  UprightFont=OfficeCodePro-Regular]
  {OfficeCodePro}
\or% font 1
\setmonofont[Scale=0.7,
  ItalicFont=inconsolatalgcitalic,
  BoldFont=inconsolatalgcbold,
  BoldItalicFont=inconsolatalgcitalic,
  UprightFont=inconsolatalgc]
  {inconsolatagc}
\else% else
%
%imposto il font per listings (si possono anche indicare le varianti, i.e. italic, bold, bolditalic)
\setmonofont[Scale=0.7,
  ItalicFont=Hack-Italic,
  BoldFont=Hack-Bold,
  BoldItalicFont=Hack-BoldItalic,
  UprightFont=Hack-Regular]
  {Hack}
\fi
%
\usepackage[T1]{fontenc}
% un'alternativa a listingsutf8 è il pacchetto minted ma richiede una libreria python chiamata Pygments
\usepackage{listingsutf8}
\definecolor{verdeoliva}{rgb}{0.3,0.3,0}
\definecolor{grigio}{rgb}{0.5,0.5,0.5}
\definecolor{blumarino}{rgb}{0.0,0,0.5}
\definecolor{panna}{rgb}{0.98,0.98,0.94}
\def\lstlistingname{Listato}
\lstset{%
  %inputencoding=utf8,
  breaklines=true,
  %extendedchars=true,              % lets you use non-ASCII characters; for 8-bits encodings only, does not work with UTF-8
  %literate=%
  %       {á}{{\'a}}1
  %       {í}{{\'i}}1
  %       {é}{{\'e}}1
  %       {ý}{{\'y}}1
  %       {ú}{{\'u}}1
  %       {ó}{{\'o}}1,
  backgroundcolor=\color{panna},   % choose the background color; you must add \usepackage{color} or \usepackage{xcolor}; should come as last argument
% basicstyle=\footnotesize\ttfamily,
  basicstyle=\ttfamily,            % è selezionato all'inizio di ogni listing
  belowskip=-0.2\baselineskip,
% basicstyle=\footnotesize,        % the size of the fonts that are used for the code
  breakatwhitespace=false,         % sets if automatic breaks should only happen at whitespace
% breaklines=true,                 % sets automatic line breaking
  captionpos=b,                    % sets the caption-position to bottom
  commentstyle=\itshape\color{verdeoliva}, % comment style (nota: all'inizio del listing seleziona \ttfamily, perciò qui seleziona la variante italic)
% deletekeywords={},            % if you want to delete keywords from the given language
% escapeinside={\%*}{*)},          % if you want to add LaTeX within your code
% firstnumber=1000,                % start line enumeration with line 1000
  frame=single,	                   % adds a frame around the code
  keepspaces=true,                 % keeps spaces in text, useful for keeping indentation of code (possibly needs columns=flexible)
  keywordstyle=\bfseries\color{blue},       % keyword style
% language=Octave,                 % the language of the code
% morekeywords={*,},            % if you want to add more keywords to the set
  numbers=left,                    % where to put the line-numbers; possible values are (none, left, right)
  numbersep=5pt,                   % how far the line-numbers are from the code
  numberstyle=\tiny\color{grigio}, % the style that is used for the line-numbers
  rulecolor=\color{black},         % if not set, the frame-color may be changed on line-breaks within not-black text (e.g. comments (green here))
  showspaces=false,                % show spaces everywhere adding particular underscores; it overrides 'showstringspaces'
  showstringspaces=false,          % underline spaces within strings only
  showtabs=false,                  % show tabs within strings adding particular underscores
  stepnumber=1,                    % the step between two line-numbers. If it's 1, each line will be numbered
  stringstyle=\color{blumarino},   % string literal style
  tabsize=2,	                   % sets default tabsize to 2 spaces
  title=\lstname%                  % show the filename of files included with \lstinputlisting; also try caption instead of title
}

\input{custom.tex}
\def\cmu{\mbox{cm$^{-1}$}}
\def\half{\frac{1}{2}}

\voffset -2cm
\hoffset -2.5cm
%\marginparwidth 0cm
\textheight 22cm
\textwidth 17cm
%\oddsidemargin  0.2cm                                                                                         
%\evensidemargin 0.4cm                                                                                         
\parindent=0pt
\begin{document}
\pagestyle{empty}

\begin{center}
{\Large \bf  Laboratorio di Calcolo per Fisici, Terza esercitazione\\[2mm]}
{\large Canale \canale, Docente: \docente}
\end{center}
\vspace{4mm}

\begin{mdframed}[backgroundcolor=panna]
  Lo scopo della quarta esercitazione di laboratorio \`e di fare pratica con
  le istruzioni di controllo di flusso \texttt{if...(then)...else}; \texttt{while...do}, le funzioni di generazione di numeri pseudo-casuali (\textit{random}) e gli array, scrivendo un programma che simula il gioco della roulette.
  \end{mdframed}
%\vspace{1mm}
%
%



\hrule
\vspace{2mm}
\textbf{$\RHD$ Prima parte:}

Creare una cartella  chiamata EX4 che conterr\`{a} il materiale di questa esercitazione.
Scrivere un programma \texttt{lancio.c} che, utilizzando opportunamente le
funzioni di generazione di numeri casuali, simuli una mano di una partita di
roulette (lancio e risultato). Il programma deve:
\begin{enumerate}
\item Generare un numero casuale intero $X$ compreso tra 1 e 36 (incluso).
\item Riconoscere se il numero generato $X$ \`e pari (E) o dispari (O) e minore
o uguale di 18 (M) oppure maggiore di 18 (P).
\item Stampare il risultato nel formato: \texttt{$X$ E/O M/P}.
\end{enumerate}

\hrule
\vspace{2mm}
\textbf{$\RHD$ Seconda parte:}
Partendo dal programma precedente, scrivere un programma chiamato \texttt{roulette.c} che
invece che un singolo lancio simuli pi\`u lanci ($N \ge 100$).
\begin{enumerate}
\item Alla fine degli $N$ lanci, stampare sullo schermo la {\em frequenza\/}
con cui si \`e verificato ciascun risultato (E/O/M/P): la frequenza
\`e quella attesa o si discosta dal valore aspettato? Verificare come
varia la frequenza al variare del parametro $N$ (cioè del numero di lanci).
\item Costruire un {\em array\/} di 36 elementi che conti quante volte è uscito ciascun risultato tra 1 e 36.
\item Scrivere il contenuto dell'array su di un file
  \texttt{isto.dat} che contenga
su due colonne i numeri da uno a 36 ({\em bin\/}) e le relative occorrenze.
Questi dati serviranno per costruire un {\em istogramma\/} dei risultati.
Il file pu\`o essere creato a mano oppure automaticamente, redirigendo l'output del programma
su un file con il comando \texttt{./roulette.x > isto.dat}.
\end{enumerate}


\hrule
\vspace{2mm}
\textbf{$\RHD$ Terza parte}
Far girare il programma \texttt{roulette.c} con un numero variabile di lanci,
ed effettuare un'analisi dell'andamento dell'istogramma dei risultati.
\begin{enumerate}
\item Creare con \texttt{python} un istogramma della distribuzione dei lanci per un numero $N$ fissato di lanci ($N \ge 100$) e salvarlo sul file su un file \texttt{isto1.png}.
  Per creare un istogramma con \texttt{python} a partire da un file \texttt{isto.dat} contenente due colonne (la prima
  contenente il numero del bin e la seconda il numero di occorrenze)
si usa il comando \lstinline[language=Python]!plt.bar(x,y,fill=True)!.
\item Generare un certo numero di file di nome \texttt{istoN.dat} che contengano l'istogramma dei risultati generati dal programma \texttt{roulette.c}, per numeri di lanci diversi, ad esempio per valori di $N$ pari a 10, 100, 10000, 100000.
 Per confrontare risultati ottenuti con un numero diverso di lanci, invece del numero di occorrenze il file dovr\`a contenere
la {\em frequenza\/} con cui si \`e verificato ciascun risultato.
\item Formattando opportunamente il grafico, paragonate gli istogrammi ottenuti per diversi valori di $N$. Che cosa si
pu\`o dire sulla distribuzione dei risultati in funzione di $N$? Il risultato \`e corerente con le vostre aspettative? Se sì/no, perch\'e? Salvare il grafico su un nuovo file (\texttt{isto2.png}) e scrivere le risposte sul file \texttt{risposte.txt}.
\end{enumerate}


\hrule
\vspace{2mm}
\textbf{$\RHD$ Quarta parte (facoltativa)}
A partire dal programma \texttt{roulette.c} creare un nuovo programma \texttt{gioco.c} che simuli una vera
partita di roulette.
Una partita si svolge come segue:
\begin{enumerate}
\item All'inizio del gioco, il giocatore riceve una dotazione iniziale di 100 euro.
\item Ad ogni mano, il giocatore  pu\`o decidere di fare una puntata  su pari o dispari, o su un numero maggiore o minore di 18 (Manque/Passe). In caso di vincita, il giocatore riceve due volte la posta per le puntate Pari/Dispari o Manque/Passe.
\item Il gioco termina dopo un numero fissato di mani (10 o 20) o quando il giocatore esaurisce il credito a sua disposizione.
\end{enumerate}

Il vostro programma dovr\`a simulare tutte le fasi della partita; a ogni mano dovr\`a chiedere al giocatore di effettuare una puntata, verificare l'eventuale vittoria e controllare se il giocatore abbia ancora soldi a propria disposizione per giocare una nuova mano. Alla fine della partita, il programma dovr\`a stampare un messaggio riassuntivo che riporti il numero totale delle mani giocate, l'ammontare totale delle puntate e il credito a disposizione del giocatore.


\vspace{4mm}

\begin{mdframed}[backgroundcolor=panna]
\textbf{Generazione di numeri casuali in C:}

Le librerie standard del C dispongono di diverse funzioni per la generazione di numeri casuali.
La funzione \texttt{drand48()} di \texttt{stdlib.h} restituisce un numero intero casuale compreso tra $[0,1)$ \@
\\
\textbf{$\RHD$} Per generare un numero {\bf razionale} casuale {\bf nell'intervallo $[0, N)$} si moltiplica \texttt{drand48()} per $N$:\@
\\
\lstinline[language=c]!x = drand48() * N;!
\\
\textbf{$\RHD$} Per generare un numero {\bf razionale} casuale {\bf nell'intervallo $[{\rm M}, N+{\rm M})$} si moltiplica per $N$ e si somma $M$ al risultato:\@
\\
\lstinline[language=c]!x = drand48() * N + M;!
\\
\textbf{$\RHD$} Per generare un numero {\bf intero} casuale {\bf nell'intervallo $[0, N]$ } si usa:\@
\\
\lstinline[language=c]!x = (int)(drand48() * (N + 1));!
\\L'istruzione \texttt{(int)x} esegue il {\em casting\/} di x prendendo solo la parte intera del prodotto. Si richiede il casting solo se x non \`{e} di tipo int. Anche in questo caso basta aggiungere la variabile $M$ per spostare l'intervallo dei numeri casuali (interi) tra $[M, N + M]$.\@
\\
{\bf NB!} Prima di chiamare la funzione \texttt{drand48()} all'interno di un programma bisogna inizializzare il {\em seme\/} (\textit{seed}) della sequenza di numeri casuali attraverso la funzione \texttt{srand48(seme)}, dove \texttt{seme} \`e un numero intero. Due sequenze inizializzate con lo stesso seme produrranno risultati identici. Se si vuole evitare che la sequenza di numeri casuali si ripeta in modo sempre uguale, si pu\`o inizializzare il seme utilizzando la funzione \texttt{time()} della libreria
nativa \texttt{time.h} inserendo all'inizio del programma, prima della chiamata della funzione \texttt{drand48()}, l'istruzione: \texttt{srand48(time(NULL));}
\end{mdframed}

\end{document}
