%! TEX program = luatex
\documentclass[11pt]{article}
\usepackage{textcomp}
\usepackage{graphicx,wasysym, mdframed,xcolor,gensymb,verbatim}
\usepackage{color}
\usepackage{floatflt}
%se si usa pdflatex:
%\usepackage[utf8]{inputenc}
%\usepackage[scaled=0.9]{FiraSans}
%
%i seguenti comandi funzionano con lualatex (si possono usare tutti i font di sistema come in vim o nel terminale!)
\usepackage{fontspec}
%\setmonofont{Inconsolatazi4}
%
% 0 OfficeCodePro 1=inconsolata 2=Hack
\ifcase 0 %font 0
\setmonofont[Scale=0.7,
  ItalicFont=OfficeCodePro-RegularItalic,
  BoldFont=OfficeCodePro-Bold,
  BoldItalicFont=OfficeCodePro-BoldItalic,
  UprightFont=OfficeCodePro-Regular]
  {OfficeCodePro}
\or% font 1
\setmonofont[Scale=0.7,
  ItalicFont=inconsolatalgcitalic,
  BoldFont=inconsolatalgcbold,
  BoldItalicFont=inconsolatalgcitalic,
  UprightFont=inconsolatalgc]
  {inconsolatagc}
\else% else
%
%imposto il font per listings (si possono anche indicare le varianti, i.e. italic, bold, bolditalic)
\setmonofont[Scale=0.7,
  ItalicFont=Hack-Italic,
  BoldFont=Hack-Bold,
  BoldItalicFont=Hack-BoldItalic,
  UprightFont=Hack-Regular]
  {Hack}
\fi
%
\usepackage[T1]{fontenc}
% un'alternativa a listingsutf8 è il pacchetto minted ma richiede una libreria python chiamata Pygments
\usepackage{listingsutf8}
\definecolor{verdeoliva}{rgb}{0.3,0.3,0}
\definecolor{grigio}{rgb}{0.5,0.5,0.5}
\definecolor{blumarino}{rgb}{0.0,0,0.5}
\definecolor{panna}{rgb}{0.98,0.98,0.94}
\def\lstlistingname{Listato}
\lstset{%
  %inputencoding=utf8,
  breaklines=true,
  %extendedchars=true,              % lets you use non-ASCII characters; for 8-bits encodings only, does not work with UTF-8
  %literate=%
  %       {á}{{\'a}}1
  %       {í}{{\'i}}1
  %       {é}{{\'e}}1
  %       {ý}{{\'y}}1
  %       {ú}{{\'u}}1
  %       {ó}{{\'o}}1,
  backgroundcolor=\color{panna},   % choose the background color; you must add \usepackage{color} or \usepackage{xcolor}; should come as last argument
% basicstyle=\footnotesize\ttfamily,
  basicstyle=\ttfamily,            % è selezionato all'inizio di ogni listing
  belowskip=-0.2\baselineskip,
% basicstyle=\footnotesize,        % the size of the fonts that are used for the code
  breakatwhitespace=false,         % sets if automatic breaks should only happen at whitespace
% breaklines=true,                 % sets automatic line breaking
  captionpos=b,                    % sets the caption-position to bottom
  commentstyle=\itshape\color{verdeoliva}, % comment style (nota: all'inizio del listing seleziona \ttfamily, perciò qui seleziona la variante italic)
% deletekeywords={},            % if you want to delete keywords from the given language
% escapeinside={\%*}{*)},          % if you want to add LaTeX within your code
% firstnumber=1000,                % start line enumeration with line 1000
  frame=single,	                   % adds a frame around the code
  keepspaces=true,                 % keeps spaces in text, useful for keeping indentation of code (possibly needs columns=flexible)
  keywordstyle=\bfseries\color{blue},       % keyword style
% language=Octave,                 % the language of the code
% morekeywords={*,},            % if you want to add more keywords to the set
  numbers=left,                    % where to put the line-numbers; possible values are (none, left, right)
  numbersep=5pt,                   % how far the line-numbers are from the code
  numberstyle=\tiny\color{grigio}, % the style that is used for the line-numbers
  rulecolor=\color{black},         % if not set, the frame-color may be changed on line-breaks within not-black text (e.g. comments (green here))
  showspaces=false,                % show spaces everywhere adding particular underscores; it overrides 'showstringspaces'
  showstringspaces=false,          % underline spaces within strings only
  showtabs=false,                  % show tabs within strings adding particular underscores
  stepnumber=1,                    % the step between two line-numbers. If it's 1, each line will be numbered
  stringstyle=\color{blumarino},   % string literal style
  tabsize=2,	                   % sets default tabsize to 2 spaces
  title=\lstname%                  % show the filename of files included with \lstinputlisting; also try caption instead of title
}

\input{custom.tex}
\usepackage[shortlabels]{enumitem}
\usepackage[italian]{babel}
\def\cmu{\mbox{cm$^{-1}$}}
\def\half{\frac{1}{2}}

\voffset -2cm
\hoffset -2.5cm
%\marginparwidth 0cm
\textheight 22cm
\textwidth 17cm
%\oddsidemargin  0.2cm                                                                                         
%\evensidemargin 0.4cm                                                                                         
\parindent=0pt

\begin{document}
\pagestyle{empty}

\begin{center}
{\Large \bf  Laboratorio di Calcolo per Fisici, Seconda esercitazione --- Addendum\\[2mm]}
{\large Canale \canale, Docente: \docente}
\end{center}
\vspace{1mm}

\begin{mdframed}[backgroundcolor=panna]
  Con questo addendum potrete continuare a familiarizzare con il linguaggio C,
  ovvero con la dichiarazione di variabili, le funzioni matematiche e semplici
  comandi per l'input/output.
\end{mdframed}
%\vspace{1mm}
%
%
\hrule
\vspace{2mm}
\textsl{Per questa esercitazione potete continuare ad utilzzare la cartella  \texttt{EX2}.} 
\vspace{0.2cm}\\
\textbf{$\RHD$ Quinta parte:} 
Un proiettile sparato da terra con velocità $v_0$ con un angolo $\alpha$ rispetto all'asse orizzontale, a distanza $x$
dal punto di lancio raggiunge la quota (da terra):
\begin{equation}
y = x \tan (\alpha )-\frac{g x^2}{2 v_0^2 {\left[\cos(\alpha)\right]}^2}   
\end{equation}
dove $g=9.82\;\hbox{m/s}^2$.
Il proiettile ricade sul terreno a una distanza dal punto di lancio $G$ chiamata gittata con:
\begin{equation}
G=\frac{2 v_0^2 \sin (\alpha ) \cos (\alpha )}{g}
\end{equation}

Tra le distanze $x_1$ e $x_2$ dal punto di lancio si trova un muro di altezza $h$.
Scrivere un programma che:
\begin{enumerate}
  \item Acquisisca da tastiera i valori di $v_0$ (in m/s), $\alpha$ (in radianti) e li stampi su schermo utilizzando per ognuno un minimo di 12 caratteri ed con un numero di cifre dopo la virgola pari a 7. 
  \item Acquisisca da tastiera un valore $L$ (in m), calcoli la quota $y$ (in m) raggiunta nel punto $x = L$ e
    calcoli la gittata $G$ (in m) del proiettile
  \item Stampi i risultati del calcolo, cioè il valore di $G$ (in m) e il valore di $y$ (in m) per $x=L$ 
        con la stessa formattazione utilizzata per $v_0$ e $\alpha$.
\end{enumerate}
Assumendo $x_1=5\,\hbox{m}$, $x_2=6\,\hbox{m}$, $h=2.05\,\hbox{m}$, $\alpha=0.7\,\hbox{rad}$ e $v_0=10\,\hbox{m/s}$,
utilizzate il programma appena scritto per rispondere alla seguente domanda: il proiettile ha urtato il muro? 
Scrivete la risposta in un file di testo chiamato \texttt{riposte\_addendum.txt}, aggiungendo anche eventuali motivazioni.
\vspace{3mm} \\
\textbf{$\RHD$ Sesta parte:} 
\begin{enumerate}
  \item Per $v_0=10$ m/s, calcolare la gittata al variare di $\alpha$ per almeno $10$ diversi valori, stampando 
   ogni volta il valore di $\alpha$ e la gittata G separati da una spazio e con 5 cifre dopo la virgola.
  
 \item Copiare i valori ottenuti al punto precedente in file di testo chiamato \texttt{gittata.dat} 
     in modo da avere due colonne, dove nella prima colonna ci saranno i valori di $\alpha$ 
      e nella seconda i corrispondenti valori della gittata $G$.

    \item Graficare il file \texttt{gittata.dat} utilizzando \texttt{python} ed unendo i punti con delle linee \\
      {\em Suggerimento:\/} per unire i punti con delle linee potete usare il seguente comando \texttt{python}:
    \vspace{0.2cm}
\begin{lstlisting}[language=Python,numbers=none]
  plt.plot(x, y, 'x-',label='Gittata')
\end{lstlisting}
 \item Per quale valore di $\alpha$ risulta massima la gittata? Vi sembra ragionevole quanto avete ottenuto?
      Scrivete la risposta ed eventuali osservazioni nel file di testo \texttt{riposte\_addendum.txt}
\end{enumerate}  
\end{document}
