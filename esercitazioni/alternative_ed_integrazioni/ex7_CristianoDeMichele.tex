%! TEX program = luatex
\documentclass[11pt]{article}
\usepackage{graphicx,wasysym, mdframed,xcolor,gensymb,verbatim}
\usepackage{color}
\usepackage{floatflt}
\usepackage[italian]{babel}
\input{custom.tex}
\def\cmu{\mbox{cm$^{-1}$}}
\def\half{\frac{1}{2}}

\voffset -2cm
\hoffset -2.5cm
%\marginparwidth 0cm
\textheight 22cm
\textwidth 17cm
%\oddsidemargin  0.2cm                                                                                         
%\evensidemargin 0.4cm                                                                                         
\parindent 0pt

\begin{document}
\pagestyle{empty}

\begin{center}
{\Large \bf  Laboratorio di Calcolo per Fisici,\\ Settima esercitazione \\[2mm]}
{\large Canale \canale, Docente: \docente}
\end{center}
\vspace{4mm}

\begin{mdframed}[backgroundcolor=gray!10]
Lo scopo della settima esercitazione di laboratorio è di fare pratica con
gli argomenti appresi durante il corso in vista delle esercitazioni valutate;\\
Vi consigliamo di utilizzarla per simulare un'esercitazione valutata: usate libri di testo e appunti,
discutete la soluzione con il vostro compagno di gruppo (a bassa voce), ma non con gli altri gruppi o con il docente.
  \end{mdframed}
%\vspace{1mm}
%
%



\hrule
\vspace{2mm}
\textbf{$\RHD$ Esercizio:}
La morra cinese \`e un antico gioco che si svolge tra due persone come segue.
Ciascuno dei giocatori porta la mano dietro la testa e, ritmicamente, la riporta
rapidamente davanti a s\'e facendo un gesto convenzionale ({\em getto}). 
I gesti possono essere di tre tipi: il pungo chiuso rappresenta un sasso, la mano aperta un foglio di carta e la mano chiusa con l'indice e il medio stesi un paio di forbici. Si confrontano i segni di ciascuno dei due giocatori e si stabilisce chi vince secondo le regole che seguono:
\begin{itemize}
\item Il sasso vince sulle forbici. 
\item Le forbici vincono sulla carta.
\item La carta vince sul sasso.
\end{itemize}
Due segni uguali conducono a un pareggio e s'ignorano. Vince chi, dopo un numero prestabilito di getti ottiene il maggior numero di vittorie. Simulare una partita di morra cinese nel modo che segue.
\begin{enumerate}
\item Tutti i codici e file relativi all'esercitazione devono essere contenuti nella cartella EX7. Il nome del programma che simula il gioco della morra cinese dovr\`{a} essere \texttt{morra.c}.
\item il programma stampa un messaggio iniziale che spiega che cosa fa
\item Il programma chiede all'utente di quanti getti sar\`a composta la partita.
L'utente inserisce questo numero attraverso la tastiera.
\item Il programma controlla che il numero di getti introdotto sia positivo e minore o uguale a 20. In caso non sia cos\`{i}, il programma stampa un messaggio di errore e chiede di nuovo in input il numero di getti.
\item Si simula il comportamento di un giocatore attraverso una funzione che
restituisce con uguale probabilit\`a un numero intero che rappresenta il segno
fatto da ciascun giocatore.
\item  A una funzione si passano in input i segni del giocatore A e del giocatore B generati tramite la funzione definita al punto precedente. La
funzione determina il vincitore in base alle regole sopra citate e restituisce
un valore che permetta di discriminare chi ha vinto il getto o se il getto \`e finito in parit\`a. La scelta della maniera in cui questo si ottiene \`e lasciata
allo studente. 
\item Per ciascun getto si determina il vincitore e si assegna un punto al vincitore (0 punti al perdente), se esiste.
\item Dopo ogni getto, il punteggio parziale dei due giocatori (ossia la somma dei punti fino a quel momento ottenuti per ciascun giocatore)  viene memorizzato in array opportuni.
\item Al termine delle partita si decreta il vincitore, stampando un messaggio su schermo assieme ai punteggi finali  dei giocatori A e B. Viene inoltre stampata sul file \texttt{punteggio.dat} la serie temporale dei punteggi dei due giocatori (ossia il punteggio parziale di ciascun giocatore calcolato getto dopo getto) salvata precedentemente.
\item Un opportuno script python crea un unico grafico \texttt{punteggio.png} contenente le serie temporali dei punteggi dei due giocatori per una partita da 20 getti.

\end{enumerate}









\end{document}
