%! TEX program = luatex
\documentclass[11pt]{article}
\usepackage{textcomp}
\usepackage{graphicx,wasysym, mdframed,xcolor,gensymb,verbatim}
\usepackage{color}
\usepackage{floatflt}
\usepackage[italian]{babel}
%se si usa pdflatex:
%\usepackage[utf8]{inputenc}
%\usepackage[scaled=0.9]{FiraSans}
%
%i seguenti comandi funzionano con lualatex (si possono usare tutti i font di sistema come in vim o nel terminale!)
\usepackage{fontspec}
%\setmonofont{Inconsolatazi4}
%
% 0 OfficeCodePro 1=inconsolata 2=Hack
\ifcase 0 %font 0
\setmonofont[Scale=0.7,
  ItalicFont=OfficeCodePro-RegularItalic,
  BoldFont=OfficeCodePro-Bold,
  BoldItalicFont=OfficeCodePro-BoldItalic,
  UprightFont=OfficeCodePro-Regular]
  {OfficeCodePro}
\or% font 1
\setmonofont[Scale=0.7,
  ItalicFont=inconsolatalgcitalic,
  BoldFont=inconsolatalgcbold,
  BoldItalicFont=inconsolatalgcitalic,
  UprightFont=inconsolatalgc]
  {inconsolatagc}
\else% else
%
%imposto il font per listings (si possono anche indicare le varianti, i.e. italic, bold, bolditalic)
\setmonofont[Scale=0.7,
  ItalicFont=Hack-Italic,
  BoldFont=Hack-Bold,
  BoldItalicFont=Hack-BoldItalic,
  UprightFont=Hack-Regular]
  {Hack}
\fi
%
\usepackage[T1]{fontenc}
% un'alternativa a listingsutf8 è il pacchetto minted ma richiede una libreria python chiamata Pygments
\usepackage{listingsutf8}
\definecolor{verdeoliva}{rgb}{0.3,0.3,0}
\definecolor{grigio}{rgb}{0.5,0.5,0.5}
\definecolor{blumarino}{rgb}{0.0,0,0.5}
\definecolor{panna}{rgb}{0.98,0.98,0.94}
\def\lstlistingname{Listato}
\lstset{%
  %inputencoding=utf8,
  breaklines=true,
  %extendedchars=true,              % lets you use non-ASCII characters; for 8-bits encodings only, does not work with UTF-8
  %literate=%
  %       {á}{{\'a}}1
  %       {í}{{\'i}}1
  %       {é}{{\'e}}1
  %       {ý}{{\'y}}1
  %       {ú}{{\'u}}1
  %       {ó}{{\'o}}1,
  backgroundcolor=\color{panna},   % choose the background color; you must add \usepackage{color} or \usepackage{xcolor}; should come as last argument
% basicstyle=\footnotesize\ttfamily,
  basicstyle=\ttfamily,            % è selezionato all'inizio di ogni listing
  belowskip=-0.2\baselineskip,
% basicstyle=\footnotesize,        % the size of the fonts that are used for the code
  breakatwhitespace=false,         % sets if automatic breaks should only happen at whitespace
% breaklines=true,                 % sets automatic line breaking
  captionpos=b,                    % sets the caption-position to bottom
  commentstyle=\itshape\color{verdeoliva}, % comment style (nota: all'inizio del listing seleziona \ttfamily, perciò qui seleziona la variante italic)
% deletekeywords={},            % if you want to delete keywords from the given language
% escapeinside={\%*}{*)},          % if you want to add LaTeX within your code
% firstnumber=1000,                % start line enumeration with line 1000
  frame=single,	                   % adds a frame around the code
  keepspaces=true,                 % keeps spaces in text, useful for keeping indentation of code (possibly needs columns=flexible)
  keywordstyle=\bfseries\color{blue},       % keyword style
% language=Octave,                 % the language of the code
% morekeywords={*,},            % if you want to add more keywords to the set
  numbers=left,                    % where to put the line-numbers; possible values are (none, left, right)
  numbersep=5pt,                   % how far the line-numbers are from the code
  numberstyle=\tiny\color{grigio}, % the style that is used for the line-numbers
  rulecolor=\color{black},         % if not set, the frame-color may be changed on line-breaks within not-black text (e.g. comments (green here))
  showspaces=false,                % show spaces everywhere adding particular underscores; it overrides 'showstringspaces'
  showstringspaces=false,          % underline spaces within strings only
  showtabs=false,                  % show tabs within strings adding particular underscores
  stepnumber=1,                    % the step between two line-numbers. If it's 1, each line will be numbered
  stringstyle=\color{blumarino},   % string literal style
  tabsize=2,	                   % sets default tabsize to 2 spaces
  title=\lstname%                  % show the filename of files included with \lstinputlisting; also try caption instead of title
}

\input{custom.tex}
\def\cmu{\mbox{cm$^{-1}$}}
\def\half{\frac{1}{2}}

\voffset -2cm
\hoffset -2.5cm
%\marginparwidth 0cm
\textheight 22cm
\textwidth 17cm
%\oddsidemargin  0.2cm                                                                                         
%\evensidemargin 0.4cm                                                                                         
\parindent=0pt

\begin{document}
\pagestyle{empty}

\begin{center}
{\Large \bf  Laboratorio di Calcolo per Fisici, Quinta esercitazione\\[2mm]}
{\large Canale \canale, Docente: \docente}
\end{center}
\vspace{4mm}

\begin{mdframed}[backgroundcolor=panna]
  Lo scopo della quinta esercitazione di laboratorio \`e di fare pratica con
le istruzioni di input/output da file e la gestione degli array, scrivendo un programma
che simula la gestione del personale di un piccolo negozio.
  \end{mdframed}
%\vspace{1mm}
%
%



\hrule
\vspace{2mm}
\textbf{$\RHD$ Prima parte:}
\begin{enumerate}
\item Creare una cartella chiamata EX5 che conterr\`{a} il materiale dell'esercitazione.
\item Creare (con un editor) un file di testo, chiamato {\em trimestre1.dat}, che 
contenga un riassunto dei dati trimestrali di vendita dei 10 commessi di
un piccolo negozio di computer. Copiate nel file le seguenti 10 righe divise in tre colonne: numero del commesso;  euro totali incassati; ore lavorate:
\begin{mdframed}[backgroundcolor=panna]
\texttt{1    450   32\\
2    322   14\\
3    870   82\\
4    765   76\\
5    198   20\\
6    145   10\\
7    828   75\\
8    932   80\\
9    267   32\\
10   424   18}
\end{mdframed}

\item Scrivere un programma \texttt{leggi.c} che, utilizzando opportunamente le funzioni di lettura da un file 
legga i dati contenuti in {\em trimestre1.dat}, li salvi in un array multidimensionale e li stampi su schermo.

\item Oltre ai dati letti dal file, il programma \texttt{leggi.c} dovrà
calcolare la {\em media oraria\/} di vendita di ciascun commesso e stamparla sullo schermo.
\end{enumerate}

\hrule
\vspace{2mm}
\textbf{$\RHD$ Seconda parte:}

Partendo dal programma precedente, scrivere un programma chiamato \texttt{dai\_voto.c} che, a partire dall'elaborazione dei risultati di
vendita trimestrale di ciascun commesso, stili una classifica dei 10 commessi e attribuisca loro
un voto. In particolare, il programma deve: 
\begin{enumerate}
\item Stilare una classifica dei commessi in base alla media oraria di vendita.
\item Dare a ciascun commesso un voto compreso tra 1 e 10 per valutare la sua performance: il commesso con la performance migliore ottiene 10 punti, il secondo 9, e così via.
\item Stampare su schermo e su un file le seguenti cinque informazioni per ciascun commesso:\\
\texttt{
$\#$  Commesso $\#$ Ore $\#$ Vendite in Euro $\#$ Media $\#$ Voto\\
}
\end{enumerate}


\hrule
\vspace{2mm}
\textbf{$\RHD$ Terza parte (Facoltativa)}
Far girare il programma \texttt{dai\_voto.c} per i quattro trimestri del 2016, utilizzando i dati riportati nel riquadro sottostante che andranno copiati in un file chiamato \texttt{anno2016.dat}. 
Il file contiene 9 colonne dove la colonna $0$ rappresenta il numero del commesso e le altre colonne sono gli euro totali incassati e le ore lavorate nei quattro trimestri. In particolare: le colonne $1$ e $2$ si riferiscono al primo trimestre, le colonne $3$ e $4$ al secondo, le colonne $5$ e $6$ al terzo e le colonne $7$ e $8$ al quarto trimestre.
Una volta ottenute le medie orarie e i voti di ciascun venditore per tutti i trimestri:
\begin{enumerate}
\item Creare con \texttt{python} un grafico che mostri l'andamento delle medie orarie per ciascun commesso; sul grafico vanno riportati simultaneamente {\bf tutti i venditori}, con una legenda esplicativa.
\item Fare un grafico simile che riporti l'andamento del voto di ciascun commesso in funzione del tempo.
\item A partire dai due grafici precedenti, qual \`{e} il commesso che ha ottenuto la performance {\em media\/} migliore su ciascun trimestre? Quale la peggiore?
Quale dei commessi ha dimostrato un margine di miglioramento e quale \`e peggiorato in maniera significativa?
Scrivere le risposte sul file \texttt{risposte.txt}.
\end{enumerate}
\begin{mdframed}[backgroundcolor=panna]
{\bf Dati Annuali di vendita (Primo, Secondo, Terzo, Quarto trimestre)}
\\
\texttt{1$\;$	450$\;$	32$\;$	450$\;$             32$\;$       360$\;$         28$\;$            500$\;$        40 \\
           2$\;$	322$\;$	14$\;$ 	380$\;$             31$\;$        415$\;$         31$\;$            324$\;$        60 \\
           3$\;$      870$\;$	82$\;$ 	900$\;$             52$\;$        320$\;$         15$\;$            400$\;$        50\\
           4$\;$      765$\;$	76$\;$ 	800$\;$             60$\;$        643$\;$         28$\;$            300$\;$        25\\
           5$\;$      198$\;$ 	20$\;$ 	600$\;$             80$\;$        472$\;$         32$\;$            120$\;$        10\\
           6$\;$      145$\;$	10$\;$ 	744$\;$             61$\;$        432$\;$         22$\;$            374$\;$        18\\ 
           7$\;$       828$\;$	75$\;$	123$\;$             12$\;$        328$\;$         19$\;$            702$\;$        40\\ 
           8$\;$       932$\;$	80$\;$	426$\;$             37$\;$        544$\;$         26$\;$            187$\;$        21\\
           9$\;$       267$\;$	32$\;$	184$\;$             14$\;$        612$\;$         43$\;$            781$\;$        72\\
           10$\;$	 424$\;$	18$\;$	212$\;$		22$\;$        374$\;$         32$\;$            643$\;$        47}
\end{mdframed}
 
 \begin{mdframed}[backgroundcolor=panna]
 {\bf Lettura da file}
 Per poter leggere i dati contenuti in un file \`{e} necessario definire nel codice un puntatore di tipo FILE. L'apertura del file avviene attraverso la funzione fopen() che vuole come parametri il nome del file e la modalita' in cui il file deve essere aperto (lettura o scrittura).\\
 \lstinline[language=c]!FILE *fpoint;!\\
 \lstinline[language=c]!fpoint = fopen("filename.dat","r");!\\
Si ricorda che nella modalit\`{a} lettura (\texttt{"r"}), il file (nell'esempio \texttt{filename.dat}) deve essere gi\`{a} esistente, altrimenti la funzione  \lstinline[language=c]!fopen! restituir\`{a}  \lstinline[language=c]!NULL! e il programma dar\`{a} un \texttt{segmentation fault} appena si prover\`{a} a leggere il contenuto del file. Si pu\`{o} dunque usare un controllo sul valore del puntatore e in caso scrivere un messaggio di errore. Per leggere i dati dal file si usa  la funzione:\\
\lstinline[language=c]!fscanf(fpoint,"%i %i....", &var1,&var2,....);! \\
Una volta letti i dati, il file deve essere chiuso con la funzione\\
\lstinline[language=c]!fclose(fpoint);!
 \end{mdframed}
 

\end{document}
