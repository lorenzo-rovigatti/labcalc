%! TEX program = luatex
\documentclass[11pt]{article}
\usepackage{textcomp}
\usepackage{graphicx,wasysym, mdframed,xcolor,gensymb,verbatim}
\usepackage{color}
\usepackage{floatflt}
\usepackage[italian]{babel}
\input{listings_styles.tex}
\newcommand{\persinfo}[3] {%
  \newcommand{\canale}{#1}
  \newcommand{\docente}{#2}
  \newcommand{\login}{#3}
}
\input{persinfo.tex}

\def\cmu{\mbox{cm$^{-1}$}}
\def\half{\frac{1}{2}}

\voffset -2cm
\hoffset -2.5cm
%\marginparwidth 0cm
\textheight 22cm
\textwidth 17cm
%\oddsidemargin  0.2cm                                                                                         
%\evensidemargin 0.4cm                                                                                         
\parindent=0pt

\begin{document}
\pagestyle{empty}

\begin{center}
{\Large \bf  Laboratorio di Calcolo per Fisici, Quinta esercitazione\\[2mm]}
{\large Canale \canale, Docente: \docente}
\end{center}
\vspace{4mm}

\begin{mdframed}[backgroundcolor=panna]
  Lo scopo della quinta esercitazione di laboratorio \`e di fare pratica con
le istruzioni di input/output da file e la gestione degli array, scrivendo un programma
che simula la gestione del personale di un piccolo negozio.
  \end{mdframed}
%\vspace{1mm}
%
%



\hrule
\vspace{2mm}
\textbf{$\RHD$ Prima parte:}
\begin{enumerate}
\item Creare una cartella chiamata EX5 che conterr\`{a} il materiale dell'esercitazione.
\item Creare (con un editor) un file di testo, chiamato {\em trimestre1.dat}, che 
contenga un riassunto dei dati trimestrali di vendita dei 10 commessi di
un piccolo negozio di computer. Copiate nel file le seguenti 10 righe divise in tre colonne: numero del commesso;  euro totali incassati; ore lavorate:
\begin{mdframed}[backgroundcolor=panna]
\texttt{1    450   32\\
2    322   14\\
3    870   82\\
4    765   76\\
5    198   20\\
6    145   10\\
7    828   75\\
8    932   80\\
9    267   32\\
10   424   18}
\end{mdframed}

\item Scrivere un programma \texttt{leggi.c} che, utilizzando opportunamente le funzioni di lettura da un file 
legga i dati contenuti in {\em trimestre1.dat}, li salvi in un array multidimensionale e li stampi su schermo.

\item Oltre ai dati letti dal file, il programma \texttt{leggi.c} dovrà
calcolare la {\em media oraria\/} di vendita di ciascun commesso e stamparla sullo schermo.
\end{enumerate}

\hrule
\vspace{2mm}
\textbf{$\RHD$ Seconda parte:}

Partendo dal programma precedente, scrivere un programma chiamato \texttt{dai\_voto.c} che, a partire dall'elaborazione dei risultati di
vendita trimestrale di ciascun commesso, stili una classifica dei 10 commessi e attribuisca loro
un voto. In particolare, il programma deve: 
\begin{enumerate}
\item Stilare una classifica dei commessi in base alla media oraria di vendita.
\item Dare a ciascun commesso un voto compreso tra 1 e 10 per valutare la sua performance: il commesso con la performance migliore ottiene 10 punti, il secondo 9, e così via.
\item Stampare su schermo e su un file le seguenti cinque informazioni per ciascun commesso:\\
\texttt{
$\#$  Commesso $\#$ Ore $\#$ Vendite in Euro $\#$ Media $\#$ Voto\\
}
\end{enumerate}


\hrule
\vspace{2mm}
\textbf{$\RHD$ Terza parte (Facoltativa)}
Far girare il programma \texttt{dai\_voto.c} per i quattro trimestri del 2016, utilizzando i dati riportati nel riquadro sottostante che andranno copiati in un file chiamato \texttt{anno2016.dat}. 
Il file contiene 9 colonne dove la colonna $0$ rappresenta il numero del commesso e le altre colonne sono gli euro totali incassati e le ore lavorate nei quattro trimestri. In particolare: le colonne $1$ e $2$ si riferiscono al primo trimestre, le colonne $3$ e $4$ al secondo, le colonne $5$ e $6$ al terzo e le colonne $7$ e $8$ al quarto trimestre.
Una volta ottenute le medie orarie e i voti di ciascun venditore per tutti i trimestri:
\begin{enumerate}
\item Creare con \texttt{python} un grafico che mostri l'andamento delle medie orarie per ciascun commesso; sul grafico vanno riportati simultaneamente {\bf tutti i venditori}, con una legenda esplicativa.
\item Fare un grafico simile che riporti l'andamento del voto di ciascun commesso in funzione del tempo.
\item A partire dai due grafici precedenti, qual \`{e} il commesso che ha ottenuto la performance {\em media\/} migliore su ciascun trimestre? Quale la peggiore?
Quale dei commessi ha dimostrato un margine di miglioramento e quale \`e peggiorato in maniera significativa?
Scrivere le risposte sul file \texttt{risposte.txt}.
\end{enumerate}
\begin{mdframed}[backgroundcolor=panna]
{\bf Dati Annuali di vendita (Primo, Secondo, Terzo, Quarto trimestre)}
\\
\texttt{1$\;$	450$\;$	32$\;$	450$\;$             32$\;$       360$\;$         28$\;$            500$\;$        40 \\
           2$\;$	322$\;$	14$\;$ 	380$\;$             31$\;$        415$\;$         31$\;$            324$\;$        60 \\
           3$\;$      870$\;$	82$\;$ 	900$\;$             52$\;$        320$\;$         15$\;$            400$\;$        50\\
           4$\;$      765$\;$	76$\;$ 	800$\;$             60$\;$        643$\;$         28$\;$            300$\;$        25\\
           5$\;$      198$\;$ 	20$\;$ 	600$\;$             80$\;$        472$\;$         32$\;$            120$\;$        10\\
           6$\;$      145$\;$	10$\;$ 	744$\;$             61$\;$        432$\;$         22$\;$            374$\;$        18\\ 
           7$\;$       828$\;$	75$\;$	123$\;$             12$\;$        328$\;$         19$\;$            702$\;$        40\\ 
           8$\;$       932$\;$	80$\;$	426$\;$             37$\;$        544$\;$         26$\;$            187$\;$        21\\
           9$\;$       267$\;$	32$\;$	184$\;$             14$\;$        612$\;$         43$\;$            781$\;$        72\\
           10$\;$	 424$\;$	18$\;$	212$\;$		22$\;$        374$\;$         32$\;$            643$\;$        47}
\end{mdframed}
 
 \begin{mdframed}[backgroundcolor=panna]
 {\bf Lettura da file}
 Per poter leggere i dati contenuti in un file \`{e} necessario definire nel codice un puntatore di tipo FILE. L'apertura del file avviene attraverso la funzione fopen() che vuole come parametri il nome del file e la modalita' in cui il file deve essere aperto (lettura o scrittura).\\
 \lstinline[language=c]!FILE *fpoint;!\\
 \lstinline[language=c]!fpoint = fopen("filename.dat","r");!\\
Si ricorda che nella modalit\`{a} lettura (\texttt{"r"}), il file (nell'esempio \texttt{filename.dat}) deve essere gi\`{a} esistente, altrimenti la funzione  \lstinline[language=c]!fopen! restituir\`{a}  \lstinline[language=c]!NULL! e il programma dar\`{a} un \texttt{segmentation fault} appena si prover\`{a} a leggere il contenuto del file. Si pu\`{o} dunque usare un controllo sul valore del puntatore e in caso scrivere un messaggio di errore. Per leggere i dati dal file si usa  la funzione:\\
\lstinline[language=c]!fscanf(fpoint,"%i %i....", &var1,&var2,....);! \\
Una volta letti i dati, il file deve essere chiuso con la funzione\\
\lstinline[language=c]!fclose(fpoint);!
 \end{mdframed}
 

\end{document}
