%! TEX program = luatex
\documentclass[11pt]{article}
\usepackage{textcomp}
\usepackage{graphicx,wasysym, mdframed,xcolor,gensymb,verbatim}
\usepackage{color}
\usepackage{floatflt}
\usepackage[italian]{babel}
\usepackage{amssymb}
\input{listings_styles.tex}
\newcommand{\persinfo}[3] {%
  \newcommand{\canale}{#1}
  \newcommand{\docente}{#2}
  \newcommand{\login}{#3}
}
\input{persinfo.tex}

\def\cmu{\mbox{cm$^{-1}$}}
\def\half{\frac{1}{2}}

\voffset -2cm
\hoffset -2.5cm
%\marginparwidth 0cm
\textheight 22cm
\textwidth 17cm
%\oddsidemargin  0.2cm                                                                                         
%\evensidemargin 0.4cm                                                                                         
\parindent=0pt
\begin{document}
\pagestyle{empty}

\begin{center}
{\Large \bf  Laboratorio di Calcolo per Fisici, Test di Autovalutazione\\[2mm]}
{\large Canale A-C, Docente: Nicoletta Gnan}
\end{center}
\vspace{4mm}

\begin{mdframed}[backgroundcolor=panna]
{\bf Nome: \qquad \qquad \qquad\qquad \qquad \qquad Cognome:}\\
\newline
{\bf Matricola:}\\
\newline 
Avete 20 minuti per rispondere alle domande. Potete usare il libro di testo e gli appunti.
\end{mdframed}
%\vspace{1mm}
%
%

\hrule
\vspace{2mm}

\begin{enumerate}
\item {\bf Si indichino quali sono le righe errate, dal punto di vista sintattico, contenute nell'estratto di codice seguente. }
\begin{lstlisting}[]
/* bla bla bla bla bla bla bla bla */
 /* bla bla bla bla bla // bla bla */
 bla bla bla */
 /* bla bla bla bla bla /* bla bla */ bla bla bla */
// bla bla bla bla bla bla bla bla
 // bla bla bla bla bla /* bla bla //
 bla bla bla //
// bla bla bla bla bla // bla bla // bla bla bla
\end{lstlisting}
\itemsep-0.1em
\item[$\square$] 1
\item[$\square$] 2
\item[$\square$] 3
\item[$\square$] 4
\item[$\square$] 5
\item[$\square$] 6
\item[$\square$] 7
\item[$\square$] 8

\item {\bf Si indichino quali sono le righe errate, dal punto di vista sintattico, contenute nell'estratto di codice seguente. Se un errore riguarda un'istruzione
suddivisa su pi\`{u} righe, si indichi solo la prima di queste righe.}
 \begin{lstlisting}[language=c]
 a = 123 + b
 c = d * 34;
 e *4 = g + f;
 h = m * 4 +
 n ;
 \end{lstlisting}
\item [\nonumber]
\item[$\square$] 1
\item[$\square$] 2
\item[$\square$] 3
\item[$\square$] 4
\item[$\square$] 5

 \item {\bf Segue un elenco di nomi di variabili. Si indichino quali sono i nomi che non possono essere usati.}
\item[$\square$]  uno
\item[$\square$] uno$-$due$-$tre
\item[$\square$]  uno$\_$due$\_$tre
\item[$\square$]  uno.due.tre
\item[$\square$]  uno2tre
\item[$\square$] 1due3

 \item {\bf Cosa fanno le istruzioni seguenti?}
 
 \begin{lstlisting}[language=c]
 printf ("Ciao");
 printf ("mondo!");
  \end{lstlisting}
 
\item[$\square$] Visualizzano la scritta {\bf Ciao mondo!} senza interruzioni di riga.
\item[$\square$] Visualizzano la scritta {\bf Ciao mondo!} riportando, alla fine, il cursore all'inizio della riga successiva.
\item[$\square$] Visualizzano la scritta {\bf Ciaomondo!} senza interruzioni di riga.
\item[$\square$] Visualizzano la scritta {\bf Ciaomondo!} riportando, alla fine, il cursore all'inizio della riga successiva.
\item[$\square$] Visualizzano la scritta {\bf Ciao}, {\bf mondo!}, in due righe separate.
 
 
 \item  {\bf A cosa serve la sequenza $\backslash n$ che appare nell'esempio seguente?}
 \begin{lstlisting}[language=c]
 printf ("bla bla\n bla bla");
 \end{lstlisting}
 
\item[$\square$]  A mandare a capo il testo in quel punto.
\item[$\square$]  A introdurre una tabulazione orizzontale.
\item[$\square$]  A impedire che in quel punto il testo sia lasciato andare a capo.


\item [\nonumber]
\item {\bf Si indichino quali sono le righe errate, contenute nell'estratto di codice seguente.}
\item [\nonumber]
\item [\nonumber]

 \begin{lstlisting}[language=c]
 int pi=1000;
 double pd=0.211;
 printf ("parte intera: %i parte decimale: 0.211", pi);
 printf ("parte intera: 1000 parte decimale: %2.4lf\n", pi, pd);
 printf ("parte intera: %i parte decimale: %lf\n", 1000, 0.211);
 printf ("parte intera: %i parte decimale: %lf\n", pi, pd, 11);
 \end{lstlisting}
 
\item[$\square$] 1
\item[$\square$] 2
\item[$\square$] 3
\item[$\square$] 4
\item[$\square$] 5
\item[$\square$] 6
\item [\nonumber]
\item{\bf Cosa rappresenta una variabile scalare di tipo 'long'?}

\item[$\square$] Un carattere, per cui non conta sapere se il segno viene considerato o meno.
\item[$\square$] Un carattere con segno.
\item[$\square$] Un carattere senza segno.
\item[$\square$] Un intero pi\`{u} breve di 'int', con segno.
\item[$\square$] Un intero pi\`{u} breve di 'int', senza segno.
\item[$\square$] Un intero normale con segno.
\item[$\square$] Un intero normale senza segno.
\item[$\square$] Un intero pi\`{u} ampio di 'int', con segno.
\item[$\square$] Un intero pi\`{u} ampio di 'int', senza segno.
\item[$\square$] Un numero in virgola mobile a precisione singola.
\item[$\square$] Un numero in virgola mobile a precisione doppia.
\item [\nonumber]
 \item {\bf Come si definisce una variabile scalare di tipo intero normale senza segno?}
 
\item[$\square$] char
\item[$\square$] unsigned char
\item[$\square$] short int
\item[$\square$] unsigned short int
\item[$\square$] int
\item[$\square$] unsigned int
\item[$\square$] long int
\item[$\square$] unsigned long int
\item[$\square$] long long int
\item[$\square$] unsigned long long int
\item[$\square$] float
\item[$\square$] double
\item[$\square$] long double
\item [\nonumber]
\item {\bf Cosa individua il tipo 'void'?}
\item[$\square$]  Una variabile qualunque.
\item[$\square$]  Una variabile composta da una quantit\`{a} infinita di byte.
\item[$\square$]  Uno spazio in memoria di lunghezza nulla.
\item[$\square$]  Uno spazio in memoria di lunghezza infinita.
\item[$\square$]  Un array di lunghezza variabile.
\item [\nonumber]
\item{\bf Indicare che tipo di numero genera l'espressione '87.45/32'}

\item[$\square$] intero
\item[$\square$] in virgola mobile

\item [\nonumber]
\item {\bf Indicare il tipo del risultato prodotto dall'espressione 'a =123.45/45', sapendo che la variabile a \`{e} di tipo 'long int'}

\item[$\square$] char
\item[$\square$] unsigned char
\item[$\square$] short int
\item[$\square$] unsigned short int
\item[$\square$] int
\item[$\square$] unsigned int
\item[$\square$] long int
\item[$\square$] unsigned long int
\item[$\square$] long long int
\item[$\square$] unsigned long long int
\item[$\square$] float
\item[$\square$] double
\item[$\square$] long double

\item [\nonumber]
 \item {\bf Osservando la porzione di codice indicata, si scriva il valore contenuto nella variabile b. Si indichi il valore in base dieci.}
 
 \begin{lstlisting}[language=c]
 int a = 4;
 int b; 
 b = a++;
 ++b;
 \end{lstlisting}
 Risultato: 
 \item [\nonumber]
 \item{\bf Osservando la porzione di codice indicata, si scriva il valore contenuto nella variabile b. Si indichi il valore in base dieci.}
 
  \begin{lstlisting}[language=c]
  int a = 4;
  int b = a % 2;
  b = b + a;
 \end{lstlisting}
  Risultato: 
  \item [\nonumber]
  \item {\bf Osservando la porzione di codice indicata, si scriva il valore contenuto nella variabile c. Si indichi il valore in base dieci.}
 
  \begin{lstlisting}[language=c]
  int a = 7, b=4, c;
  c = b > a;
   \end{lstlisting}
  Risultato: 
  \item [\nonumber]
  \item {\bf Osservando il programma indicato, cosa viene visualizzato?}
  
  \begin{lstlisting}[language=c]
   #include <stdio.h>
    int main (void){
    int x = 4;
    if (x % 2){
    	printf ("Sono felice :-)\n");
    }else{
    	printf ("Sono triste :-(\n");
     }
  }
 \end{lstlisting}   
 
\item[$\square$] Sono felice :-)
\item[$\square$] Sono triste :-(
\item[$\square$] Non so rispondere
\item [\nonumber]
\item {\bf Osservando il programma indicato, cosa contiene la variabile k alla fine del ciclo?}

 \begin{lstlisting}[language=c]
#include <stdio.h>
int main (void){
 int j = 7;
 int k = 7; 
 while (j > 2){
  	j--;
   	k++;
  }
  printf ("La variabile k contiene %i\n", k);
 }
\end{lstlisting}   

Risultato: 
\item [\nonumber]
\item {\bf Osservando il programma indicato, cosa contiene la variabile k alla fine del ciclo?}
\begin{lstlisting}[language=c]
#include <stdio.h>
 int main (void){
 int j;
 int k = 7;
 for (j = 2; j <=7; j++){
 	k++;
 }
 printf ("La variabile k contiene %i\n", k);
 }
\end{lstlisting} 

Risultato: 
\item [\nonumber]
\item{\bf Indica al posto dei puntini l'operazione logica corrispondente alla tabella di verit\`{a} indicata.}
\begin{center}
\includegraphics[width=0.28\textwidth]{mm11_connettivi}
\end{center}


\item [\nonumber]
\item {\bf A cosa serve il programma descritto dal seguente diagramma di flusso?}
\begin{center}
\includegraphics[width=0.45\textwidth]{mm4_fig5.jpg}
\end{center}

\item[$\square$]a dare in ouput i numeri da 0 a 50 a ritroso e di 5 in 5
\item[$\square$]a dare in ouput 10 volte il numero 50
\item[$\square$]a dare in ouput i numeri da 0 a 50

\item [\nonumber]
\item {\bf Che differenza esiste tra programma sorgente (cio\`{e} un programma scritto in linguaggio c) ed un programma oggetto?}

\item[$\square$]Il programma sorgente \`{e} l'algoritmo tradotto in un linguaggio ad alto livello mentre il programma oggetto \`{e} tradotto in linguaggio macchina
\item[$\square$]Entrambi i programmi sono tradotti in linguaggio macchina
\item[$\square$]Il programma sorgente \`{e} tradotto in linguaggio macchina mentre il programma oggetto \`{e} tradotto in linguaggio ad alto livello

\item [\nonumber]
\item {\bf Quali sono gli input del linker?}
\item[$\square$] i file oggetto
\item[$\square$] i file eseguibili
\item[$\square$] i file sorgente

\end{enumerate}



 

\end{document}
