%! TEX program = luatex
\documentclass[9pt]{article}
\usepackage{textcomp}
\usepackage{graphicx,wasysym, mdframed,xcolor,gensymb,verbatim}
\usepackage{color}
\usepackage{floatflt}
\usepackage[italian]{babel}
\usepackage{amssymb}
%se si usa pdflatex:
%\usepackage[utf8]{inputenc}
%\usepackage[scaled=0.9]{FiraSans}
%
%i seguenti comandi funzionano con lualatex (si possono usare tutti i font di sistema come in vim o nel terminale!)
\usepackage{fontspec}
%\setmonofont{Inconsolatazi4}
%
% 0 OfficeCodePro 1=inconsolata 2=Hack
\ifcase 0 %font 0
\setmonofont[Scale=0.7,
  ItalicFont=OfficeCodePro-RegularItalic,
  BoldFont=OfficeCodePro-Bold,
  BoldItalicFont=OfficeCodePro-BoldItalic,
  UprightFont=OfficeCodePro-Regular]
  {OfficeCodePro}
\or% font 1
\setmonofont[Scale=0.7,
  ItalicFont=inconsolatalgcitalic,
  BoldFont=inconsolatalgcbold,
  BoldItalicFont=inconsolatalgcitalic,
  UprightFont=inconsolatalgc]
  {inconsolatagc}
\else% else
%
%imposto il font per listings (si possono anche indicare le varianti, i.e. italic, bold, bolditalic)
\setmonofont[Scale=0.7,
  ItalicFont=Hack-Italic,
  BoldFont=Hack-Bold,
  BoldItalicFont=Hack-BoldItalic,
  UprightFont=Hack-Regular]
  {Hack}
\fi
%
\usepackage[T1]{fontenc}
% un'alternativa a listingsutf8 è il pacchetto minted ma richiede una libreria python chiamata Pygments
\usepackage{listingsutf8}
\definecolor{verdeoliva}{rgb}{0.3,0.3,0}
\definecolor{grigio}{rgb}{0.5,0.5,0.5}
\definecolor{blumarino}{rgb}{0.0,0,0.5}
\definecolor{panna}{rgb}{0.98,0.98,0.94}
\def\lstlistingname{Listato}
\lstset{%
  %inputencoding=utf8,
  breaklines=true,
  %extendedchars=true,              % lets you use non-ASCII characters; for 8-bits encodings only, does not work with UTF-8
  %literate=%
  %       {á}{{\'a}}1
  %       {í}{{\'i}}1
  %       {é}{{\'e}}1
  %       {ý}{{\'y}}1
  %       {ú}{{\'u}}1
  %       {ó}{{\'o}}1,
  backgroundcolor=\color{panna},   % choose the background color; you must add \usepackage{color} or \usepackage{xcolor}; should come as last argument
% basicstyle=\footnotesize\ttfamily,
  basicstyle=\ttfamily,            % è selezionato all'inizio di ogni listing
  belowskip=-0.2\baselineskip,
% basicstyle=\footnotesize,        % the size of the fonts that are used for the code
  breakatwhitespace=false,         % sets if automatic breaks should only happen at whitespace
% breaklines=true,                 % sets automatic line breaking
  captionpos=b,                    % sets the caption-position to bottom
  commentstyle=\itshape\color{verdeoliva}, % comment style (nota: all'inizio del listing seleziona \ttfamily, perciò qui seleziona la variante italic)
% deletekeywords={},            % if you want to delete keywords from the given language
% escapeinside={\%*}{*)},          % if you want to add LaTeX within your code
% firstnumber=1000,                % start line enumeration with line 1000
  frame=single,	                   % adds a frame around the code
  keepspaces=true,                 % keeps spaces in text, useful for keeping indentation of code (possibly needs columns=flexible)
  keywordstyle=\bfseries\color{blue},       % keyword style
% language=Octave,                 % the language of the code
% morekeywords={*,},            % if you want to add more keywords to the set
  numbers=left,                    % where to put the line-numbers; possible values are (none, left, right)
  numbersep=5pt,                   % how far the line-numbers are from the code
  numberstyle=\tiny\color{grigio}, % the style that is used for the line-numbers
  rulecolor=\color{black},         % if not set, the frame-color may be changed on line-breaks within not-black text (e.g. comments (green here))
  showspaces=false,                % show spaces everywhere adding particular underscores; it overrides 'showstringspaces'
  showstringspaces=false,          % underline spaces within strings only
  showtabs=false,                  % show tabs within strings adding particular underscores
  stepnumber=1,                    % the step between two line-numbers. If it's 1, each line will be numbered
  stringstyle=\color{blumarino},   % string literal style
  tabsize=2,	                   % sets default tabsize to 2 spaces
  title=\lstname%                  % show the filename of files included with \lstinputlisting; also try caption instead of title
}

\input{custom.tex}
\def\cmu{\mbox{cm$^{-1}$}}
\def\half{\frac{1}{2}}

\voffset -2cm
\hoffset -2.5cm
%\marginparwidth 0cm
\textheight 22cm
\textwidth 17cm
%\oddsidemargin  0.2cm                                                                                         
%\evensidemargin 0.4cm                                                                                         
\parindent=0pt
\begin{document}
\pagestyle{empty}

\begin{center}
{\Large \bf  Laboratorio di Calcolo per Fisici, Test di Autovalutazione\\[2mm]}
{\large Canale \canale, Docente: \docente}
\end{center}
\vspace{4mm}
%\def\AC{A-C}
%\ifx\cmacro\AC%if nicoletta print also following mdframed
\begin{mdframed}[backgroundcolor=panna]
{\bf Nome: \qquad \qquad \qquad\qquad \qquad \qquad Cognome:}\\
\newline
{\bf Matricola:}\\
%\newline 
Avete 20 minuti per rispondere alle domande. Potete usare il libro di testo e gli appunti.
\end{mdframed}
%\fi
%\vspace{1mm}
%
%

\hrule
\vspace{2mm}

\begin{enumerate}
\item {\bf Si indichino quali sono le righe errate, dal punto di vista sintattico, contenute nell'estratto di codice seguente. }
\begin{lstlisting}[]
/* bla bla bla bla bla bla bla bla */
 /* bla bla bla bla bla // bla bla */
 bla bla bla */
 /* bla bla bla bla bla /* bla bla */ bla bla bla */
// bla bla bla bla bla bla bla bla
 // bla bla bla bla bla /* bla bla //
 bla bla bla //
// bla bla bla bla bla // bla bla // bla bla bla
\end{lstlisting}
\itemsep-0.1em
\item[$\square$] 1
\item[$\square$] 2
\item[$\square$] 3
\item[$\square$] 4
\item[$\square$] 5
\item[$\square$] 6
\item[$\square$] 7
\item[$\square$] 8

\item {\bf Si indichino quali sono le righe errate, dal punto di vista sintattico, contenute nell'estratto di codice seguente. Se un errore riguarda un'istruzione
suddivisa su pi\`{u} righe, si indichi solo la prima di queste righe.}
 \begin{lstlisting}[language=c]
 a = 123 + b
 c = d * 34;
 e *4 = g + f;
 h = m * 4 +
 n ;
 \end{lstlisting}
\item [\nonumber]
\item [\nonumber]
\item[$\square$] 1
\item[$\square$] 2
\item[$\square$] 3
\item[$\square$] 4
\item[$\square$] 5

 \item {\bf Segue un elenco di nomi di variabili. Si indichino quali sono i nomi che non possono essere usati.}
\item[$\square$]  uno
\item[$\square$] uno$-$due$-$tre
\item[$\square$]  uno$\_$due$\_$tre
\item[$\square$]  uno.due.tre
\item[$\square$]  uno2tre
\item[$\square$] 1due3

 \item {\bf Cosa fanno le istruzioni seguenti?}
 
 \begin{lstlisting}[language=c]
 printf ("Ciao");
 printf ("mondo!");
  \end{lstlisting}
 
\item[$\square$] Visualizzano la scritta {\bf Ciao mondo!} senza interruzioni di riga.
\item[$\square$] Visualizzano la scritta {\bf Ciao mondo!} riportando, alla fine, il cursore all'inizio della riga successiva.
\item[$\square$] Visualizzano la scritta {\bf Ciaomondo!} senza interruzioni di riga.
\item[$\square$] Visualizzano la scritta {\bf Ciaomondo!} riportando, alla fine, il cursore all'inizio della riga successiva.
\item[$\square$] Visualizzano la scritta {\bf Ciao}, {\bf mondo!}, in due righe separate.
 
 
 \item  {\bf A cosa serve la sequenza $\backslash n$ che appare nell'esempio seguente?}
 \begin{lstlisting}[language=c]
 printf ("bla bla\n bla bla");
 \end{lstlisting}
 
\item[$\square$]  A mandare a capo il testo in quel punto.
\item[$\square$]  A introdurre una tabulazione orizzontale.
\item[$\square$]  A impedire che in quel punto il testo sia lasciato andare a capo.


\item [\nonumber]
\item {\bf Si indichino quali sono le righe errate, contenute nell'estratto di codice seguente.}
\item [\nonumber]
\item [\nonumber]

 \begin{lstlisting}[language=c]
 int pi=1000;
 double pd=0.211;
 printf ("parte intera: %i parte decimale: 0.211", pi);
 printf ("parte intera: 1000 parte decimale: %2.4lf\n", pi, pd);
 printf ("parte intera: %i parte decimale: %lf\n", 1000, 0.211);
 printf ("parte intera: %i parte decimale: %lf\n", pi, pd, 11);
 \end{lstlisting}
 
\item[$\square$] 1
\item[$\square$] 2
\item[$\square$] 3
\item[$\square$] 4
\item[$\square$] 5
\item[$\square$] 6
\item [\nonumber]
\item{\bf A cosa ci si riferisce con il termine 'long'? Barrare una o pi\`{u} caselle}

\item[$\square$] Ad un carattere, per cui non conta sapere se il segno viene considerato o meno.
\item[$\square$] Ad un carattere con segno.
\item[$\square$] Ad un carattere senza segno.
\item[$\square$] Ad un intero pi\`{u} breve di 'int', con segno.
\item[$\square$] Ad un intero pi\`{u} breve di 'int', senza segno.
\item[$\square$] Ad un intero normale con segno.
\item[$\square$] Ad un intero normale senza segno.
\item[$\square$] Ad un intero pi\`{u} ampio di 'int', con segno.
\item[$\square$] Ad un intero pi\`{u} ampio di 'int', senza segno.
\item[$\square$] Ad un numero in virgola mobile a precisione singola.
\item[$\square$] Ad un numero in virgola mobile a precisione doppia.
\item[$\square$] Ad un modificatore di tipo.
\item [\nonumber]
 \item {\bf Come si definisce una variabile scalare di tipo intero normale senza segno?}
 
\item[$\square$] char
\item[$\square$] unsigned char
\item[$\square$] short int
\item[$\square$] unsigned short int
\item[$\square$] int
\item[$\square$] unsigned int
\item[$\square$] long int
\item[$\square$] unsigned long int
\item[$\square$] long long int
\item[$\square$] unsigned long long int
\item[$\square$] float
\item[$\square$] double
\item[$\square$] long double
\item [\nonumber]

\item {\bf L'espressione "Se x>0" si traduce in c come:}
\item[$\square$]  if x>0;
\item[$\square$]  if x>0
\item[$\square$]  If(x>0)
\item[$\square$]  If[x>0]
\item[$\square$]  if\{x>0\};
\item[$\square$]  if(x>0)

\item [\nonumber]
\item {\bf Nel costrutto IF, il blocco di pi\`{u} istruzioni \`{e} racchiuso:}
\item[$\square$]  Tra parentesi quadre
\item[$\square$]  Tra parentesi tonde
\item[$\square$]  Tra parentesi graffe 
\item[$\square$]  Non sono necessarie le parentesi
\item [\nonumber]

\item {\bf Nel ciclo WHILE la condizione viene valutata:}
\item[$\square$]  Alla fine dell'esecuzione del blocco di istruzioni
\item[$\square$]  Prima di eseguire il blocco di istruzioni
\item[$\square$]  Nel blocco di istruzioni, tra un'istruzione e un'altra; 

\item [\nonumber]
\item {\bf Un ciclo DO-WHILE  esegue il blocco fino a che la condizione \`{e}:}
\item [$\square$] vera
\item [$\square$] falsa

\item [\nonumber]
\item{\bf Indicare che tipo di numero genera l'espressione '87.45/32'}

\item[$\square$] intero
\item[$\square$] in virgola mobile

\item [\nonumber]
\item {\bf Indicare il tipo del risultato prodotto dall'espressione 'a =123.45/45', sapendo che la variabile a \`{e} di tipo 'long int'}

\item[$\square$] unsigned char
\item[$\square$] int
\item[$\square$] unsigned int
\item[$\square$] long int
\item[$\square$] unsigned long int
\item[$\square$] long long int

\item [\nonumber]
 \item {\bf Osservando la porzione di codice indicata, si scriva il valore contenuto nella variabile b. Si indichi il valore in base dieci.}
  \begin{lstlisting}[language=c]
 int a = 4;
 int b; 
 b = a++;
 ++b;
 \end{lstlisting}
 Risultato: 
 \item [\nonumber]
 \item{\bf Osservando la porzione di codice indicata, si scriva il valore contenuto nella variabile b. Si indichi il valore in base dieci.}
 
  \begin{lstlisting}[language=c]
  int a = 4;
  int b = a % 2;
  b = b + a;
 \end{lstlisting}
  Risultato: 
  \item [\nonumber]
  \item {\bf Osservando la porzione di codice indicata, si scriva il valore contenuto nella variabile c. Si indichi il valore in base dieci.}
 
  \begin{lstlisting}[language=c]
  int a = 7, b=4, c;
  c = b > a;
  \end{lstlisting}
  Risultato: 
  \item [\nonumber]
  \item {\bf Osservando il programma indicato, cosa viene visualizzato?}
  
  \begin{lstlisting}[language=c]
   #include <stdio.h>
    int main (void){
    int x = 4;
    if (x % 2){
    	printf ("Sono felice :-)\n");
    }else{
    	printf ("Sono triste :-(\n");
     }
  }
 \end{lstlisting}   
 
\item[$\square$] Sono felice :-)
\item[$\square$] Sono triste :-(
\item[$\square$] Non so rispondere
\item [\nonumber]
\item {\bf Osservando il programma indicato, cosa contiene la variabile k alla fine del ciclo?}

 \begin{lstlisting}[language=c]
#include <stdio.h>
int main (void){
 int j = 7;
 int k = 7; 
 while (j > 2){
  	j--;
   	k++;
  }
  printf ("La variabile k contiene %i\n", k);
 }
\end{lstlisting}   

Risultato: 
\item [\nonumber]
\item {\bf Osservando il programma indicato, cosa contiene la variabile k alla fine del ciclo?}
\begin{lstlisting}[language=c]
#include <stdio.h>
 int main (void){
 int j;
 int k = 7;
 for (j = 2; j <=7; j++){
 	k++;
 }
 printf ("La variabile k contiene %i\n", k);
 }
\end{lstlisting} 

Risultato: 
\item [\nonumber]
\item{\bf Indica al posto dei puntini l'operazione logica corrispondente alla tabella di verit\`{a} indicata.}
\begin{center}
\includegraphics[width=0.18\textwidth]{mm11_connettivi}
\end{center}


\item [\nonumber]
\item {\bf A cosa serve il programma descritto dal seguente diagramma di flusso?}
\begin{center}
\includegraphics[width=0.25\textwidth]{mm4_fig5.jpg}
\end{center}

\item[$\square$]a dare in ouput i numeri da 0 a 50 a ritroso e di 5 in 5
\item[$\square$]a dare in ouput 10 volte il numero 50
\item[$\square$]a dare in ouput i numeri da 0 a 50

\item [\nonumber]
\item {\bf Che differenza esiste tra codice sorgente ed codice oggetto?}

\item[$\square$]Dal codice sorgente si ottiene il codice oggetto mentre il codice oggetto \`{e} un programma in un formato intermedio non portabile ottenuto dal sorgente da cui verrà prodotto il codice finale in linguaggio macchina. 
\item[$\square$]Nessuna sono la stessa cosa.
\item[$\square$]Il codice sorgente \`{e} tradotto in linguaggio macchina mentre il codice oggetto \`{e} tradotto in linguaggio ad alto livello

\item [\nonumber]
\item {\bf Quali sono gli input del linker?}
\item[$\square$] i file oggetto
\item[$\square$] i file eseguibili
\item[$\square$] i file sorgente

\end{enumerate}



 

\end{document}
